\documentclass[notheorems,mathserif,table,compress]{beamer}  %dvipdfm选项是关键,否则编译统统通不过
%%------------------------常用宏包------------------------
%%注意, beamer 会默认使用下列宏包: amsthm, graphicx, hyperref, color, xcolor, 等等
\usepackage{fontspec,xunicode,xltxtra}  % for XeTeX
%%------------------------ThemeColorFont------------------------
%% Presentation Themes
% \usetheme[<options>]{<name list>}
\usetheme{Madrid}
%% Inner Themes
% \useinnertheme[<options>]{<name>}
%% Outer Themes
% \useoutertheme[<options>]{<name>}
\useoutertheme{miniframes} 
%% Color Themes 
% \usecolortheme[<options>]{<name list>}
%% Font Themes
% \usefonttheme[<options>]{<name>}
\setbeamertemplate{background canvas}[vertical shading][bottom=white,top=structure.fg!7] %%背景色, 上25%的蓝, 过渡到下白.
\setbeamertemplate{theorems}[numbered]
\setbeamertemplate{navigation symbols}{}   %% 去掉页面下方默认的导航条.
\usepackage{zhfontcfg}
%\setsansfont[Mapping=tex-text]{文泉驿正黑}  %% 需要fontspec宏包
     %如果装了Adobe Acrobat,可在font.conf中配置Adobe字体的路径以使用其中文字体
     %也可直接使用系统中的中文字体如SimSun,SimHei,微软雅黑 等
     %原来beamer用的字体是sans family;注意Mapping的大小写,不能写错
     %设置字体时也可以直接用字体名,以下三种方式等同:
     %\setromanfont[BoldFont={黑体}]{宋体}
     %\setromanfont[BoldFont={SimHei}]{SimSun}
     %\setromanfont[BoldFont={"[simhei.ttf]"}]{"[simsun.ttc]"}
%%------------------------MISC------------------------
\graphicspath{{figures/}}         %% 图片路径. 本文的图片都放在这个文件夹里了.
%%------------------------正文------------------------
\begin{document}
\XeTeXlinebreaklocale "zh"         % 表示用中文的断行
\XeTeXlinebreakskip = 0pt plus 1pt % 多一点调整的空间
%%----------------------------------------------------------
%% This is only inserted into the PDF information catalog. Can be left
%% out.
%%%
%% Delete this, if you do not want the table of contents to pop up at
%% the beginning of each subsection:
\AtBeginSection[]{                              % 在每个Section前都会加入的Frame
  \frame<handout:0>{
    \frametitle{内容提要}\small
    \tableofcontents[current,currentsubsection]
  }
}
\AtBeginSubsection[]                            % 在每个子段落之前
{
  \frame<handout:0>                             % handout:0 表示只在手稿中出现
  {
    \frametitle{下一节内容}\small
    \tableofcontents[current,currentsubsection] % 显示在目录中加亮的当前章节
  }
}
%%----------------------------------------------------------
\title[Numerical Analysis]{Numerical Analysis}
\subtitle{Mathematics of Scientific Computing}
\author[qiu]{主讲人~~~~~\textcolor{olive}{朱亚菲}\\
    \quad 幻灯片制作~~\textcolor{olive}{孙晓庆}}
\institute[中国海洋大学]{\small\textcolor{violet}{中国海洋大学~~信息科学与工程学院}}
\date{2013~年~9~月~6~日}
%\titlegraphic{\vspace{-6em}\includegraphics[height=7cm]{ouc}\vspace{-6em}}
\frame{ \titlepage }
%%----------------------------------------------------------
\section*{目录}
\frame{\frametitle{目录}\tableofcontents}
%%----------------------------------------------------------
\section{Order of convergence and Additional Basic  concepts}
\subsection{Convergent Sequences} %如果你想书签不出现问题,请不要用\XeTeX
                                 %这类复杂的指令,直接写XeTeX吧
\begin{frame}
  \frametitle{Convergent Sequences}
  \begin{itemize}
  \item In case of real numbers, a computer program may produce a sequence of real numbers $x_1$, $x_2$, $x_3$,... that are approaching the correct answer.
  \item if there corresponds to each positive $\varepsilon$ a real numbers r such that $|x_{n}-l|< \varepsilon$ whenever $n>r$ 
  \item For example  
\begin{displaymath} 
\lim_{n \to \infty}\frac{n+1}{n}=1
\end{displaymath}
  \end{itemize}
  % \XeTeXpicfile "./logo.jpg" xscaled 100 yscaled 100 %插图也没有问题
\end{frame}
\subsection{Orders of Convergence} 
\begin{frame}
  \frametitle{Orders of Convergence}
  \begin{itemize}
  \item Let $[ x_n ]$ be a sequence of real numbers tending to a limit $x^*$ . We say
that the rate of convergence is at least \textbf{linear} if there are a constant$ c < 1$ and an
integer N such that   
\begin{displaymath}
 |x_{n+1}-x^*|\leq  c|x_n-x^*|   (n\geq  N)
\end{displaymath}

  \item   We say that the rate of convergence is at least \textbf{superlinear} if there exists a sequence
$\varepsilon_{n}$ tending to 0 and an integer N such that  
\begin{displaymath}
|x_{n+1}-x^*|\leq  \varepsilon_n|x_n-x^*|  (n\geq  N)
\end{displaymath}
  \end{itemize}
  
\end{frame}

\begin{frame}
  \frametitle{Orders of Convergence}
 \begin{itemize}
\item The convergence is at least \textbf{quadratic} if there are a constant C (not necessarily less
than 1) and an integer N such that    
\begin{displaymath}
|x_{n+1}-x^*|\leq  C|x_n-x^*|^2  (n\geq  N)
\end{displaymath}
 
\item In general, if there are positive constants C and α and an integer N such that 
\begin{displaymath}
 |x_{n+1}-x^*|\leq  C|x_n-x^*|^\alpha  (n\geq  N)
\end{displaymath}
 We say that the rate of convergence is of \textbf{order $\alpha$}  at least.
  \end{itemize}

\end{frame}



\subsection{Big O and Little o }
\begin{frame}
  \frametitle{Big O and Little o Notation}
  \begin{itemize}
  \item Let $[x_n]$ and $[\alpha_n]$ be two different  sequences , We write $x_n=O(\alpha_n)$
 if there are constants C and $n_0$ such that $|x_n| \leq C|\alpha_n|$ when $n\geq n_0$ .$x_n$ is "big oh"of $[\alpha_n]$   
\newline

  \item  The  equation $x_n=o(\alpha_n)$ .For some $\varepsilon$ have   $\varepsilon \rightarrow 0$ and $|x_n|\leq \varepsilon_n|\alpha_n|$ .   
   \begin{displaymath}
  (\lim_{n \to \infty}\frac{x_n}{\alpha_n}=0) 
  \end{displaymath} 
  \end{itemize}
\end{frame}

\begin{frame}
  \frametitle{Big O and Little o Notation}
  \begin{itemize}
  \item In  general, we write $f(x)=O(g(x))$  $(x\rightarrow x^*)$\\
  when  there  is a constant  C and a   neighborhood of $x^*$ such  that  $|f(x)|\leq C|g(x)|$ in that neighborhood.
\newline

 \item  Similarly, $f(x)=o(g(x))$ $(x\rightarrow x^*)$ means that 
\begin{displaymath}
 (\lim_{x \to x^*}\frac {f(x)}{g(x)}=0)
\end{displaymath} 
\end{itemize}
\end{frame}

\subsection{Mean-Value Theorem  for Integrals }
\begin{frame}
  \frametitle{Theorem}
  \begin{itemize}
 \item Let u and v be continuous real-valued functions on an interval [a,b],\\
and suppose  that $v \geq  0$ .Then  there  exists a point $\xi$ in $[a,b]$such that 
\begin{displaymath}
\int _{a}^{b} u(x)v(x)dx {=} u(\xi)\int_{a}^{b}v(x)dx
\end{displaymath} 
  \end{itemize}
\end{frame}
\subsection{Nested Multiplication}

\begin{frame}
  \frametitle{Nested Multiplication}
  \begin{itemize}
 \item A polynomials can be rewritten in a nested  form  that it  requires  only a few more  than  the minimum  number of multiplications  when  evaluating it.
 \item The polynomial $p(x)=a_nx^n+a_{n-1}x^{n-1}+....+a_2x^2+a_1x+a_0$  
 \item can  be written  using  standard  mathematical  notation involving  the  sum $\sum$ and  product $\prod$ as follows:
 \begin{displaymath}
p(x)=\sum_{k=0}^n a_kx^k 
\end{displaymath}
  \item To evaluate  the  polynomial  efficiently, we can group  the  terms  using  nested  multiplication:$p(x)=a_0+x(a_1+x(a_2+....+x(a_{n-1}+xa_n))...))$
\end{itemize}
\end{frame}

%\begin{frame}
 % \frametitle{Nested Multiplication}
 % \begin{itemize}
%  \item  This  corresponds to  the  following  simple  algorithm  involving  only  n  multiplications  and n additions:
%\end{itemize}
%\begin{plaintext}
%$p\leftarrow a_n$
%for $k=n-1$ to 0 step -1 do
%$p\leftarrow xp+a_k$
%end  do
%fjeij
%\end{plaintext}
%\end{frame}

\subsection{Least Upper Bound Axiom}
\begin{frame}
  \frametitle{Least Upper Bound Axiom}
  \begin{description}
 \item Definition  of  supremum

      The supremum of S is v$(v=supS=lubS)$ if  and only if

     \begin{enumerate}
       \item v is  an upper bound for S and
        \item no real  number smaller than v is  an upper bound for S
    \end{enumerate}
 %\end{description}
 
 %\begin{description}
 \item Definition  of  infilmum 

      The infilmum of S is u$(u=supS=lubS)$ if  and only if

    \begin{enumerate}
     \item  u is a lower bound for S and
      \item no  real  number  greater  than  u is  a lower bound for S
     \end{enumerate}
\end{description}
\end{frame}

\subsection{Explicit and Implicit Functions}
\begin{frame}
  \frametitle{Explicit Functions}
  \begin{itemize}
\item A explicit function is usually defind via an explicit formula,from which a value of function can be computed for each  argument.
\item  for  example  $f(x)=\sqrt{7x^3-2x}$
\item A function $y=f(x)$ is well defined  by  the following  differential equation  with  an  initial  condition 
\begin{displaymath}
\begin{array}{ll}
y'=1+\sin y\\
 y(0)=0\\
\end{array} %\right.
\end{displaymath}
\end{itemize}
\end{frame}

\begin{frame}
  \frametitle{Implicit Functions}
  \begin{itemize}
\item Let G be a function of two real variables defined and continuously differentiable  in a neighborhood of $(x_0,y_0)$ ,\\
if $G(x_0,y_0)=0$ and$\partial G/\partial y\neq 0$ at $(x_0,y_0)$, then there is a positive $\delta$ and a continuously  differentiable function f defined  for $|x-x_0|<\delta$ such that $f(x_0)=y_0$ and  $G(x,f(x))=0$ for $|x-x_0|<\delta$ 


\end{itemize}
\end{frame}
\end{document}
