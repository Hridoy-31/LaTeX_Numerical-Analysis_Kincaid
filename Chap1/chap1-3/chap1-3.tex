\documentclass[notheorems,mathserif,table,compress]{beamer}  %dvipdfm选项是关键,否则编译统统通不过
%%------------------------常用宏包------------------------
%%注意, beamer 会默认使用下列宏包: amsthm, graphicx, hyperref, color, xcolor, 等等
\usepackage{fontspec,xunicode,xltxtra}  % for XeTeX
%%------------------------ThemeColorFont------------------------
%% Presentation Themes
% \usetheme[<options>]{<name list>}
\usetheme{Madrid}
%% Inner Themes
% \useinnertheme[<options>]{<name>}
%% Outer Themes
% \useoutertheme[<options>]{<name>}
\useoutertheme{miniframes} 
%% Color Themes 
% \usecolortheme[<options>]{<name list>}
%% Font Themes
% \usefonttheme[<options>]{<name>}
\setbeamertemplate{background canvas}[vertical shading][bottom=white,top=structure.fg!7] %%背景色, 上25%的蓝, 过渡到下白.
\setbeamertemplate{theorems}[numbered]
\setbeamertemplate{navigation symbols}{}   %% 去掉页面下方默认的导航条.
\usepackage{zhfontcfg}
%\setsansfont[Mapping=tex-text]{文泉驿正黑}  %% 需要fontspec宏包
     %如果装了Adobe Acrobat,可在font.conf中配置Adobe字体的路径以使用其中文字体
     %也可直接使用系统中的中文字体如SimSun,SimHei,微软雅黑 等
     %原来beamer用的字体是sans family;注意Mapping的大小写,不能写错
     %设置字体时也可以直接用字体名,以下三种方式等同:
     %\setromanfont[BoldFont={黑体}]{宋体}
     %\setromanfont[BoldFont={SimHei}]{SimSun}
     %\setromanfont[BoldFont={"[simhei.ttf]"}]{"[simsun.ttc]"}
%%------------------------MISC------------------------
\graphicspath{{figures/}}         %% 图片路径. 本文的图片都放在这个文件夹里了.
%%------------------------正文------------------------
\begin{document}
\XeTeXlinebreaklocale "zh"         % 表示用中文的断行
\XeTeXlinebreakskip = 0pt plus 1pt % 多一点调整的空间
%%----------------------------------------------------------
%% This is only inserted into the PDF information catalog. Can be left
%% out.
%%%
%% Delete this, if you do not want the table of contents to pop up at
%% the beginning of each subsection:
\AtBeginSection[]{                              % 在每个Section前都会加入的Frame
  \frame<handout:0>{
    \frametitle{内容提要}\small
    \tableofcontents[current,currentsubsection]
  }
}
\AtBeginSubsection[]                            % 在每个子段落之前
{
  \frame<handout:0>                             % handout:0 表示只在手稿中出现
  {
    \frametitle{下一节内容}\small
    \tableofcontents[current,currentsubsection] % 显示在目录中加亮的当前章节
  }
}
%%----------------------------------------------------------
\title[Numerical Analysis]{Numerical Analysis}
\subtitle{Mathematics of Scientific Computing}
\author[sun]{主讲人~~~~~\textcolor{olive}{孙晓庆}\\
    \quad 幻灯片制作~~\textcolor{olive}{孙晓庆}}
\institute[中国海洋大学]{\small\textcolor{violet}{中国海洋大学~~信息科学与工程学院}}
\date{2013~年~9~月~13~日}
%\titlegraphic{\vspace{-6em}\includegraphics[height=7cm]{ouc}\vspace{-6em}}
\frame{ \titlepage }
%%----------------------------------------------------------
\section*{目录}
\frame{\frametitle{目录}\tableofcontents}
%%----------------------------------------------------------
\section{Difference Equations}

\subsection{Basic  Concepts} %如果你想书签不出现问题,请不要用\XeTeX
                                 %这类复杂的指令,直接写XeTeX吧
\begin{frame}
\frametitle{Basic  Concepts}
\begin{itemize}
\item V stand for  the set of all infinite  sequences of complex numbers,\\
such  as  $x=[x_{1}, x_{2}, x_{3}, ...]$   $y=[y_{1}, y_{2}, y_{3}, ...]$,  A sequence is a complex-valued  function  defined  on  the  set of positive  integers $N=\{1,2,3,...\}$. $x(n)$ is the value of the function $x$ at  the  argument n. 
\item In the  set V, we define two operations:
\begin{displaymath} 
x+y=[{x_{1}+y_{1}}, {x_{2}+y_{2}}, {x_{3}+y_{3}},...]
\end{displaymath}
\begin{displaymath}
{\lambda x} =[{\lambda x_{1}}, {\lambda x_{2}}, {\lambda x_{3}},...]
\end{displaymath}
\begin{displaymath}
(x+y)_n=x_n+y_n
\end{displaymath}
\begin{displaymath}
(\lambda x)_n=\lambda x_n
\end{displaymath}
with  the adoption of these definitions, V becomes a vector space. 
\item $0$ element in V, namely,$0=[0,0,0,...]$
\end{itemize}
  % \XeTeXpicfile "./logo.jpg" xscaled 100 yscaled 100 %插图也没有问题
\end{frame}

%\subsection{basic  concepts} 

\begin{frame}
\frametitle{Basic  Concepts}
\begin{itemize}
\item  \textbf{Shift operator} or \textbf{Displacement operator} denoted by E and  defined by the equation 
\begin{displaymath}
Ex=[x_{2}, x_{3}, x_{4},\ldots]\quad where\quad x=[x_{1}, x_{2}, x_{3},\ldots]
\end{displaymath}
\begin{displaymath}
(Ex)_n =x_{n+1} 
\end{displaymath}
\begin{displaymath}
(EEx)_n = x_{n+2}
\end{displaymath}
\begin{displaymath}
({E^k}x)_n = x_{n+k}
\end{displaymath}
\item \textbf{Linear differece operator} can be expressed  as linear combinations of powers of E. The general form  of  such an operator is
\begin{displaymath}
L=\sum_{i=0}^m c_{i}E^{i}  
\end{displaymath}
\end{itemize}
\end{frame}


\begin{frame}
\frametitle{Basic Concepts}
\begin{itemize}
\item  $E^0$ is defined  as the identity operator, 
\begin{displaymath}
(E^{0}x)_{n}=x_{n}
\end{displaymath}
\item  The linear difference operator make up a linear subspace in the set of all linear  operators from V to V. The powers of E provide a basis  for this  subspace.
\item we could write  $L=p(E)$. \\
Where $p$ is a polynomial called the \textbf{characteristic polynomial} of L and defined by
\begin{displaymath}
p(\lambda)=\sum_{i=0}^m c_{i}\lambda^{i}  
\end{displaymath}
\item The set $\{x:Lx=0\}$ is a linear subspace of V; it is called the  \textbf{null space} of L. 
\end{itemize}
\end{frame}

%\subsection{basic  concepts }

\begin{frame}
\frametitle{Basic  Concepts}
\begin{itemize}
\item Let us  consider a concrete  example of L , say by  taking $c_0=2$, $c_1={-3}$, $c_2=1$,and  all other $c_i=0$. The resulting equation is a linear difference equation,can be written  in  three forms:
\begin{displaymath}%\begin{eqnarray*}  
{(E^{2}-3E^{1}+2E^{0})}x=0 
\end{displaymath}
\begin{displaymath}
 x_{n+2}-3x_{n+1}+2x_{n}=0 \qquad (n\geq 1)  
\end{displaymath}
\begin{displaymath}
p(E)x=0 \qquad     p(\lambda)=\lambda^{2}-3\lambda +2
\end{displaymath}
\end{itemize}
\end{frame}


\begin{frame}
\frametitle{Basic  Concepts }
\begin{itemize}
\item It is easy to generate sequences that solve. we can choose $x_{1}$ and $x_{2}$ arbitarily and then determine $x_{3}$,$x_{4}$,$...$ we can obtain the solutions  
\begin{displaymath}
[1, 0,{-2}, {-6},{-14},{-30},\ldots] 
\end{displaymath}
\begin{displaymath}
[1, 1, 1, 1,\ldots]   
\end{displaymath}
\begin{displaymath}
[2, 4, 8, 16,\ldots]
\end{displaymath}
\item Two solutions are obviously of the form $x_n=\lambda^{n}$, with $\lambda=1$ or $\lambda=2$. Putting the $x_{n}=\lambda^{n}$ in the yields.
\begin{displaymath}
\lambda^{n+2}-3\lambda^{n+1}+2\lambda^{n}=0
\end{displaymath}
\begin{displaymath}
\lambda^{n}(\lambda-1)(\lambda-2)=0
\end{displaymath} 
\end{itemize}
\end{frame}


\begin{frame}
\begin{itemize}
\item The other solution of the type sought-namely, $[0,0,0,...]$, this we call the  \textbf{trivial solution}. The solutions $u_{n}=1$ and $v_{n}=2^{n}$
are a basis for the solution space.
\item  Let $x$ be any solution, $x=\alpha u+\beta v$, the equation means that $x_{n}={\alpha u_{n}}+{\beta v_{n}}$ for all $n$.
\item For $n=1$ and $n=2$, we have 
\begin{displaymath}
x_{1}=\alpha+2\beta
\end{displaymath}
\begin{displaymath}
x_{2}=\alpha+4\beta
\end{displaymath} 
\item because the determinant of the matrix is not $0$,Equation uniquely determines $\alpha$ and $\beta$.
\end{itemize}
\end{frame}



\begin{frame}
\begin{itemize}
\item If this equation is true for indices less than $n$, then it is true for $n$ because 
\begin{displaymath}
x_{n}=3x_{n-1}-2x_{n-2}=3(\alpha u_{n-1}+\beta v_{n-1})-2(\alpha u_{n-2}+\beta v_{n-2})=\alpha u_{n}+\beta v_{n}
\end{displaymath}
\item this example illustrates the case of simple roots of the characteristic polynomial. 
\end{itemize}
\end{frame}

\subsection{Simple Roots}
\begin{frame}
\frametitle{Theorem on Null Space}
\begin{itemize}
\item 	If $p$ is a polynomial and $\lambda$ is a root of $p$, then one solution of the difference equation $p(E)x=0$ is $[\lambda, \lambda^{2}, \lambda^{3},...]$. If all the root of $p$ are simple and nonzero, then each solution of the difference equation is a linear combination of such special solutions.
\end{itemize}  
\end{frame}


\subsection{Multiple Roots}
\begin{frame}
\frametitle{Multiple Roots}
\begin{itemize}
\item When $p$ has multiple roots, slove a difference equation $p(E)x=0$. Define $x(\lambda)=[\lambda, \lambda^{2}, \lambda^{3}, ...]$. If $p$ is any polynomial, we have seen that
\begin{displaymath}
p(E)x(\lambda)=p(\lambda)x(\lambda)
\end{displaymath}
A differentiation with respect to $\lambda$ yields
\begin{displaymath}
p(E)x'(\lambda)=p'(\lambda)x(\lambda)+p(\lambda)x'(\lambda)
\end{displaymath}
If $\lambda$ is a multiple root of $p$, then $p(\lambda)=p'(\lambda)=0$, $x(\lambda)$ and $x'(\lambda)$ are solutions of the difference equence.
\end{itemize}
\end{frame}

\begin{frame}
\frametitle{Multiple Roots}
\begin{itemize}
\item If $\lambda$ is a root of $p$ having multiplicity $k$, then the following sequences are solutions of the difference equation $p(E)x=0$:
\begin{displaymath}
x(\lambda)=[\lambda, \lambda^{2}, \lambda^{3},...]
\end{displaymath}
\begin{displaymath}
x'(\lambda)=[1, 2\lambda, 3\lambda^2,...]
\end{displaymath}
\begin{displaymath}
x''(\lambda)=[0, 2, 6\lambda,...]
\end{displaymath}
\begin{displaymath}
    .
    .
    .
\end{displaymath}
\begin{displaymath}
x^{(k-1)}(\lambda)=\frac{d^{(k-1)}}{d\lambda^{(k-1)}}[\lambda, \lambda^{2}, \lambda^{3},...]
\end{displaymath}
   \end{itemize}
\end{frame}

\begin{frame}
\frametitle{Theorem on Basic for Null Space}
\begin{itemize}
\item Let $p$ be a polynomial satisfying $p(0)\neq0$. Then a basis for the null space of $p(E)$ is obtained as follows: With each root $\lambda$ of $p$ having multiplicity $k$, associate the basic solutions $x(\lambda)$, $x'(\lambda)$,...,$x^{(k-1)}(\lambda)$, where $x(\lambda)=[\lambda, \lambda^{2}, \lambda^{3},...]$
\end{itemize}
\end{frame}


\begin{frame}
\frametitle{EXAMPLE 1}
\begin{itemize}
\item Determine the general solution of this difference equation: $4x_{n}+7x_{n-1}+2x_{n-2}-3x_{n-3}=0$
\item Solution $p(\lambda)=4\lambda^{3}+7\lambda^{2}+2\lambda-1=(\lambda+1)^{2}(4\lambda-1)$ \\
$p$ has a double root at $-1$ and a simple root at $\frac{1}{4}$. The basic solutions are
\begin{displaymath}
x(-1)=[-1, 1, -1, 1,...]
\end{displaymath}
\begin{displaymath}
x'(-1)=[1, -2, 3, -4,...]
\end{displaymath}
\begin{displaymath}
x(\frac{1}{4})=[\frac{1}{4}, \frac{1}{16}, \frac{1}{64}, ...]
\end{displaymath}
\item The general solution is $x_{n}=\lambda(-1)^n+\beta n(-1)^{(n-1)}+\gamma (\frac{1}{4})^n$
\end{itemize}
\end{frame}

\subsection{Stable Difference Equations}
\begin{frame}
\frametitle{Stable Difference Equations}
\begin{itemize}
\item An element $x=[x_{1}, x_{2},...]$ of V is said to be bounded if there is a constant $c$ such that $|x_{n}|\leq c$ for all $n$. A difference equation of the form $p(E)x=0$ is said to be \textbf{stable} if all of its solutions are bounded.
\newline
\item Theorem on Stable Difference Equations \\
For a polynomial p satisfying $p(0)\neq0$, these properties are equivalent:\\
1.The difference equation $p(E)x=0$ is stable.\\
2.All roots of satisfy $|z|\leq1$, and all multiple roots satisfy $|z|<1$
\end{itemize}
\end{frame}

\begin{frame}
\frametitle{EXAMPLE 2}
\begin{itemize}
\item Determine whether this difference equation is stable:
\begin{displaymath}
4x_{n}+7x_{n-1}+2x_{n-2}-x_{n-3}=0
\end{displaymath}
\item Solution The given equation is of the form $p(E)x=0$, where $p(\lambda)=4\lambda^{3}+7\lambda^{2}+2\lambda-1$.\\
p has a double root at $-1$ and a simple root at $\frac{1}{4}$. The equation is therefore unstable.
\end{itemize}
\end{frame}
  



\end{document}
