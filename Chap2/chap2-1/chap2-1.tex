\documentclass[notheorems,mathserif,table,compress]{beamer}  %dvipdfm选项是关键,否则编译统统通不过
%%------------------------常用宏包------------------------
%%注意, beamer 会默认使用下列宏包: amsthm, graphicx, hyperref, color, xcolor, 等等
\usepackage{fontspec,xunicode,xltxtra}  % for XeTeX
\usepackage{verbatim}
\usepackage{mathabx}
%%------------------------ThemeColorFont------------------------
%% Presentation Themes
% \usetheme[<options>]{<name list>}
\usetheme{Madrid}
%% Inner Themes双精度计算
% \useinnertheme[<options>]{<name>}
%% Outer Themes
% \useoutertheme[<options>]{<name>}
\useoutertheme{miniframes} 
%% Color Themes 
% \usecolortheme[<options>]{<name list>}
%% Font Themes
\usefonttheme{serif}
\setbeamertemplate{background canvas}[vertical shading][bottom=white,top=structure.fg!7] %%背景色, 上25%的蓝, 过渡到下白.
\setbeamertemplate{theorems}[numbered]
\setbeamertemplate{navigation symbols}{}   %% 去掉页面下方默认的导航条.
\usepackage{zhfontcfg}
%\setsansfont[Mapping=tex-text]{文泉驿正黑}  %% 需要fontspec宏包
     %如果装了Adobe Acrobat,可在font.conf中配置Adobe字体的路径以使用其中文字体
     %也可直接使用系统中的中文字体如SimSun,SimHei,微软雅黑 等
     %原来beamer用的字体是sans family;注意Mapping的大小写,不能写错
     %设置字体时也可以直接用字体名,以下三种方式等同:
     %\setromanfont[BoldFont={黑体}]{宋体}
     %\setromanfont[BoldFont={SimHei}]{SimSun}
     %\setromanfont[BoldFont={"[simhei.ttf]"}]{"[simsun.ttc]"}
%%------------------------MISC------------------------
\graphicspath{{figures/}}         %% 图片路径. 本文的图片都放在这个文件夹里了.
%%------------------------正文------------------------
\begin{document}
\XeTeXlinebreaklocale "zh"         % 表示用中文的断行
\XeTeXlinebreakskip = 0pt plus 1pt % 多一点调整的空间
%%----------------------------------------------------------
%% This is only inserted into the PDF information catalog. Can be left
%% out.
%%%
%% Delete this, if you do not want the table of contents to pop up at
%% the beginning of each subsection:
\AtBeginSection[]{                              % 在每个Section前都会加入的Frame
  \frame<handout:0>{
    \frametitle{Contents}\small
    \tableofcontents[current,currentsubsection]
  }
}

\AtBeginSubsection[]                            % 在每个子段落之前
{
  \frame<handout:0>                             % handout:0 表示只在手稿中出现
  {
    \frametitle{Contents}\small
    \tableofcontents[current,currentsubsection] % 显示在目录中加亮的当前章节
  }
}

%%----------------------------------------------------------
\title[Numerical Analysis]{Numerical Analysis}
\subtitle{Chapter Two: Computer Arithmetic}
\author[zhu]{主讲人~~~~~\textcolor{olive}{朱亚菲}\\
    \quad 幻灯片制作~~\textcolor{olive}{朱亚菲}}
\institute[中国海洋大学]{\small\textcolor{violet}{中国海洋大学~~信息科学与工程学院}}
\date{2013~年~9~月~13~日}
%\titlegraphic{\vspace{-6em}\includegraphics[height=7cm]{ouc}\vspace{-6em}}
\frame{ \titlepage }
%%----------------------------------------------------------
\section*{Contents}
\frame{\frametitle{Contents}\tableofcontents}



%%----------------------------------------------------------
\section{2.0 Introduction}

\begin{frame}
  \frametitle{2.0 Introduction}
  \begin{itemize}
  \item In this  chapter, we are going to:
  \begin{itemize}
  \item Describe the floating-point number system;
  \item Develop basic facts about roundoff errors, which may contaminate computer calculations;
  \item Discuss other types of errors and loss of significance;
  \item Survey some stable/unstable algorithms and ill-conditioned problems.
  \end{itemize}
  \end{itemize}
\end{frame}


\begin{frame}
  \frametitle{2.0 Introduction}
  Most computers deal with real numbers in the binary number system, in contrast to the decimal number system that humans prefer to use. The   binary system uses 2 as the base in the same way that the decimal system uses 10. \\
  In the binary system, only the two digits 0 and 1 are used. A typical number in the binary system can be written out in detail; for example, 
  \begin{displaymath}
  (1001.11101)_2 = 1 \times 2^3 +0 \times 2^2 +0 \times 2^1 + 1 \times 2^0 + 1 \times 2^{-1} + 1 \times 2^{-2} + 
  \end{displaymath}
  \begin{displaymath}
  1 \times 2^{-3} + 0 \times 2^{-4} + 1 \times 2^{-5}
  \end{displaymath}
\end{frame}


\begin{frame}
  \frametitle{2.0 Introduction}
  Since the typical computer works internally in the binary system but communicates with its human users in the decimal system, conversion procedures must be executed by the computer. These come into use at input and output time. Ordinarily the user need not be concerned with these conversions, but they do involve small roundoff errors.
\end{frame}


\begin{frame}
  \frametitle{2.0 Introduction}
  Computers are not able to operate using real numbers expressed with more than a fixed number of digits. Even a simple number like 1/10 cannot be stored exactly in any binary machines. It requires an infinite binary expression:
  \begin{displaymath}
  \frac{1}{10} =(0.00011001100110011\ldots)_2
  \end{displaymath}
\end{frame}


\begin{frame}
  \frametitle{2.0 Introduction}
  For example, if we read 0.1 into a 32-bit computer workstation and then print it out to 40 decimal places, we obtain the following results:
  \begin{displaymath}
  0.1000000014901161193847656250000000000000
  \end{displaymath}
  Usually, we won't notice this conversion error since printing using the default fotmat would show us 0.1.
\end{frame}


\section{2.1 Floating-point Numbers and Roundoff Errors}

\subsection{2.1.1 Rounding}

%如果你想书签不出现问题,请不要用\XeTeX
                                 %这类复杂的指令,直接写XeTeX吧

\begin{frame}
  \frametitle{2.1.1 Rounding}
  Consider a positive decimal number x of the form $0.\square \square \square \ldots \square \square \square$ with m digits to the right of the decimal point.

  One rounds x to n desimal places (n<m) in a manner that depends on the value of the (n+1)st digit.

  If this digit is 0, 1, 2, 3, or 4, then the nth digit is not changed
and all following digits are discarded.

  If it is a 5, 6, 7, 8, or 9,then the nth digit is increased by one unit and the remaining digits are discarded.
\end{frame}


\begin{frame}
  \frametitle{2.1.1 Rounding}
  Here are some examples of seven-digit numbers being correctly rounded to four digits:
  \begin{displaymath}
  0.1735 \longleftarrow 0.1735499
  \end{displaymath}
  \begin{displaymath}
  1.000 \longleftarrow 0.9999500
  \end{displaymath}
  \begin{displaymath}
  0.4322 \longleftarrow 0.4321609
  \end{displaymath}
\end{frame}


\begin{frame}
  \frametitle{2.1.1 Rounding}
  If x is rounded so that $\widetilde{x}$ is the n-digit approximation to it, then 
  \begin{displaymath}
  |x- \widetilde{x}| \leq \frac{1}{2} \times 10^{-n}
  \end{displaymath}
  Proof:
  \begin{itemize}
     \item If the (n+1)st digit of x is 0, 1, 2, 3, or 4, then $x= \widetilde{x} + \varepsilon$ with $\varepsilon<\frac{1}{2}\times 10^{-n}$ and the Inequality follows.
     \item If it is 5, 6, 7, 8, or 9, then $\widetilde{x}=\widehat{x}+10^{-n}$ where $\widehat{x}$ is a number with the same n digits as x and all digits beyond the nth are 0. Now $x=\widehat{x}+\delta \times 10^{-n}$ with $\delta \geq \frac{1}{2}$ and $\widetilde{x}-x=(1-\delta) \times 10^{-n}$. Since $1- \delta \le \frac{1}{2}$, the Inequality follows.
  \end{itemize}
\end{frame}


\begin{frame}
  \frametitle{2.1.1 Rounding}
  If $x$ is a decimal number, the chopped or truncated $n$-digit approximation to it is the number $\widehat{x}$ obtained by simply discarding all digits beyond the $n$th. For it, we have
  \begin{displaymath}
  |x-\widehat{x}|<10^{-n}
  \end{displaymath}
  The relationship between $x$ and $\widehat{x}$ is that $x-\widehat{x}$ has 0 in the first $n$ places and $x=\widehat{x}+\delta \times 10^{-n}$ and the Inequality follows.
\end{frame}



\subsection{2.1.2 Normalized Scientific Notation}

\begin{frame}
  \frametitle{2.1.2 Normalized Scientific Notation}
  We can use scientific notation in the binary system. Now we have
  \[x=\pm q \times 2^m\]
  where $\frac{1}{2} \le q < 1$(if $x \ne 0$) and m is an integer.
  %\newline
  The number $q$ is called the mantissa and the integer $m$ the exponent. Both $q$ and $m$ are base 2 numbers.
\end{frame}


\subsection{2.1.3 Hypothetical Computer Marc-32}

\begin{frame}
  \frametitle{2.1.3 Hypothetical Computer Marc-32}
  The bits composing a word in the Marc-32 are allocated in the following way when representing a nonzero real number $x=\pm q \times 2^m$:
  \begin{displaymath}
  sign\,\, of\,\, the\,\, real\,\, number\,\, x \quad 1 bit
  \end{displaymath}
  \begin{displaymath}
  biased\,\, exponent\,\, (integer\,\, e) \quad 8 bits
  \end{displaymath}
  \begin{displaymath}
  mantissa\,\, part\,\, (real\,\, number\,\, f) \quad 23 bits
  \end{displaymath}
\end{frame}


\begin{frame}
  \frametitle{2.1.3 Hypothetical Computer Marc-32}
  Nonzero normalized machine numbers are bit strings whose values are decoded as follows:
  \begin{displaymath}
  x=(-1)^s q \times 2^m
  \end{displaymath}
  where
  \begin{displaymath}
  q=(1.f)_2 \quad and \quad m=e-127
  \end{displaymath}
  Here $1 \le q < 2$ and the most significant bit in $q$ is 1 and is not explicitly stored. Also, here s is the bit representing the sign of $x$ (positive: bit 0, negative: bit 1), $m=e-127$ is the 8-bit biased exponent, and f is the 23-bit fractional part of the real number $x$ that, together with an implicit leading bit 1, yields the significant digit field $(1.\square \square \square \ldots \square \square \square)_2$. 
\end{frame}


\begin{frame}
  \frametitle{2.1.3 Hypothetical Computer Marc-32}
  The restriction that $|m|$ require no more than 8 bits means that 
  \begin{displaymath}
  0<e<(11111111)_2=2^8-1=255
  \end{displaymath}
  and the values e=0 and e=255 are reserved for special cases such as $\pm0,\pm \infty$, and NaN (Not a Number). Since $m=e-127$, we take    $-126 \le m \le 127$ and the Marc-32 can handle numbers as small as $2^{-126} \approx 1.2 \times 10^{-38}$ and as large as $(2-2^{-23})2^{127} \approx 3.4 \times 10^{38}$. This is not a sufficiently generous range of magnitudes for some scientific calculations, and for this reason and others, we occasionally must write a program in \textsf{double-precision} or \textsf{extended-precision arithmatic}.
\end{frame}


\subsection{2.1.4 Zero,Infinity,NaN}

\begin{frame}
  \frametitle{2.1.4 Zero,Infinity,NaN}
In IEEE standard arithmetic, there are two forms of 0,+0 and -0, in single precisions, represented in the computer by the words $[00000000]_{16}$ and $[80000000]_{16}$, respectively.\\
Similarly, there are two forms of infinity, $+\infty$ and $-\infty$, in single precision, represented by the computer words $[7F800000]_{16}$ and $[FF800000]_{16}$, respectively.\\
NaN means Not a Number and results from an indeterminate operation such as $0/0, \infty - \infty, x+NaN$, and so on. NaN's are represented by computer words with e=255 and $f \ne 0$.
\end{frame}


\subsection{2.1.5 Machine Rounding}

\begin{frame}
  \frametitle{2.1.5 Machine Rounding}
  The usual (default) rounding mode is round to nearest: The closer of the two machine numbers on the left and right of the real number is selected. In the case of a tie, round to even is uesd: If the real number is exactly halfway between the machine numbers to its left and right, then the even machine number is chosen.\\
  With this default rounding scheme (round to nearest plus round to even), the maximum error is half a unit in the least significant place.
\end{frame}


\subsection{2.1.6 Intrinsic Procedures in Fortran 90}

\begin{frame}
  \frametitle{2.1.6 Intrinsic Procedures in Fortran 90}
   \begin{description}
   \item[digits] number of significant (binary) digits.
   \item[epsilon] a positive number that is almost negligible compared to unity.
   \item[huge] largest number.
   \item[maxexponent] maximum (binary) exponent.
   \item[minexponent] minimum (most negative, binary) exponent.
   \item[precision] decimal precion.
   \item[radix] the base for the computer floating-point number system.
   \item[range] decimal exponent range.
   \item[tiny] smallest positive number.
   \end{description}
\end{frame}


\subsection{2.1.7 IEEE Standard Floating-Point Arithmetic}

\begin{frame}
  \frametitle{2.1.7 IEEE Standard Floating-Point Arithmetic}
  The Marc-32 representation for real numbers is patterned after the usual floating-point representation in 32-bit machines, which is the IEEE standard floating-point representation. We have chosen to give only a brief description here. For example, computers that implement floating-point arithmetic according to the current officail standard use 80 bits for internal calculations. There are many additional concerpts---guard bit, denormalized numbers, unnormalized numbers, double rounding, and others---that enter into any detailed discussion of this subjects.
\end{frame}


\subsection{2.1.8 Nearby Machine Numbers}

\begin{frame}
  \frametitle{2.1.8 Nearby Machine Numbers}
  We now want to assess the error involved in approximating a given positive real number x by a nearby machine number in the Marc-32. We assume that 
  \begin{displaymath}
  x=q\times2^m \quad 1\le q < 2 \quad -126 \le m \le 127
  \end{displaymath}
  We ask, What is the machine number closest to x ? First, we write
  \begin{displaymath}
  x=(1.a_1a_2\ldots a_{23}a_{24}a_{25}\ldots)_2\times2^m
  \end{displaymath}
  in which each $a_i$ is either 0 or 1.  
\end{frame}


\begin{frame}
  \frametitle{2.1.8 Nearby Machine Numbers}
  One nearby machine number is obtained by simply discarding the excess bits $a_{24}a_{25}\ldots$This procedure is usually called chopping. The resulting number is
  \begin{displaymath}
  x_-=(1.a_1a_2\ldots a_{23})_2\times2^m
  \end{displaymath}
  Another nearby machine number lies to the right of x. It is obtained by rounding up; that is, we drop the excess bits as before but increase the last remaining bit $a_{23}$ by one unit. This number is 
  \begin{displaymath}
  x_+=\big( (1. a_1 a_2 \ldots a_{23})_2 + 2^{-23} \big) \times 2^m
  \end{displaymath}
\end{frame}


\begin{frame}
  \frametitle{2.1.8 Nearby Machine Numbers}
  There are two situations, the closer of $x_-$ and $x_+$ is chosen to represent $x$ in the computer.\\
  If $x$ is represented better by $x_-$, then we have
  \begin{displaymath}
  |x-x_-|\le \frac{1}{2}|x_+-x_-|=\frac{1}{2}\times2^{m-23}=2^{m-24}
  \end{displaymath}
  In this case, the relative error is bounded as follows:
  \begin{displaymath}
  \left| \frac{x-x_-}{x} \right| \le \frac{2^{m-24}}{q \times 2^m} =\frac{1}{q} \times 2^{-24} \le 2^{-24}
  \end{displaymath}
  In the second case, $x$ is closer to $x_+$ than to $x_-$ and we have
  \begin{displaymath}
  |x-x_+| \le \frac{1}{2}|x_+-x_-|=2^{m-24}
  \end{displaymath}
  The same analysis then shows the relative error to be no greater than $2^{-24}$.
\end{frame}


\begin{frame}
  \frametitle{2.1.8 Nearby Machine Numbers}
  What has just been said about the Marc-32 can be summarized: If $x$ is a nonzero real number within the range of the machine, then the machine number $x^*$ closest to $x$ satisfies the inequality
  \begin{displaymath}
  \left| \frac{x-x_*}{x} \right| \le 2^{-24} 
  \end{displaymath}
  By letting $\delta =(x^*-x)/x$, we can write this inequality in the form
  \begin{displaymath}
  fl(x)=x(1+\delta) \quad |\delta| \le 2^{-24}
  \end{displaymath}
  The notation $fl(x)$ is used to denote the floating-point machine number $x^*$ closest to $x$.
  The number $2^{-24}$ that occurs in the preceding inequalities is called the unit roundoff error for the Marc-32.
\end{frame}



\subsection{2.1.9 Floating-Point Error Analysis}

\begin{frame}
  \frametitle{2.1.9 Floating-Point Error Analysis}
  We assume that the design of this machine is such that whenever two machine numbers are to be combined arithmetically, the combination is first correctly formed, then normalized, rounded off, and finally stored in memory in a machine word.  
\end{frame}


\subsection{2.1.10 Relative Error Analysis}

\begin{frame}
  \frametitle{2.1.10 Relative Error Analysis}
  \begin{description}
  \item[THEOREM 1] \textsf{Theorem on Relative Roundoff Error in Adding}\\
  Let $x_0,x_1,\ldots,x_n$ be positive machine numbers in a computer whose unit roundoff error is $\varepsilon$. Then the relative roundoff error in computing 
  \[ \sum_{i=0}^n x_i \]
  in the usual way is at most $(1+\varepsilon)^n-1 \approx n\varepsilon$.
  \end{description}
\end{frame}


\end{document}
