\documentclass[notheorems,mathserif,table,compress]{beamer}  %dvipdfm选项是关键,否则编译统统通不过
%%------------------------常用宏包------------------------
%%注意, beamer 会默认使用下列宏包: amsthm, graphicx, hyperref, color, xcolor, 等等
\usepackage{fontspec,xunicode,xltxtra}  % for XeTeX
\usepackage{verbatim}
\usepackage{mathabx}
\usepackage{iplouclistings} 
%%------------------------ThemeColorFont------------------------
%% Presentation Themes
% \usetheme[<options>]{<name list>}
\usetheme{Madrid}
%% Inner Themes双精度计算
% \useinnertheme[<options>]{<name>}
%% Outer Themes
% \useoutertheme[<options>]{<name>}
\useoutertheme{miniframes} 
%% Color Themes 
% \usecolortheme[<options>]{<name list>}
%% Font Themes
\usefonttheme{serif}
\setbeamertemplate{background canvas}[vertical shading][bottom=white,top=structure.fg!7] %%背景色, 上25%的蓝, 过渡到下白.
\setbeamertemplate{theorems}[numbered]
\setbeamertemplate{navigation symbols}{}   %% 去掉页面下方默认的导航条.
\usepackage{zhfontcfg}
%\setsansfont[Mapping=tex-text]{文泉驿正黑}  %% 需要fontspec宏包
     %如果装了Adobe Acrobat,可在font.conf中配置Adobe字体的路径以使用其中文字体
     %也可直接使用系统中的中文字体如SimSun,SimHei,微软雅黑 等
     %原来beamer用的字体是sans family;注意Mapping的大小写,不能写错
     %设置字体时也可以直接用字体名,以下三种方式等同:
     %\setromanfont[BoldFont={黑体}]{宋体}
     %\setromanfont[BoldFont={SimHei}]{SimSun}
     %\setromanfont[BoldFont={"[simhei.ttf]"}]{"[simsun.ttc]"}
%%------------------------MISC------------------------
\graphicspath{{figures/}}         %% 图片路径. 本文的图片都放在这个文件夹里了.
%%------------------------正文------------------------
\begin{document}
\XeTeXlinebreaklocale "zh"         % 表示用中文的断行
\XeTeXlinebreakskip = 0pt plus 1pt % 多一点调整的空间
%%----------------------------------------------------------
%% This is only inserted into the PDF information catalog. Can be left
%% out.
%%%
%% Delete this, if you do not want the table of contents to pop up at
%% the beginning of each subsection:
\AtBeginSection[]{                              % 在每个Section前都会加入的Frame
  \frame<handout:0>{
    \frametitle{Contents}\small
    \tableofcontents[current,currentsubsection]
  }
}

\AtBeginSubsection[]                            % 在每个子段落之前
{
  \frame<handout:0>                             % handout:0 表示只在手稿中出现
  {
    \frametitle{Contents}\small
    \tableofcontents[current,currentsubsection] % 显示在目录中加亮的当前章节
  }
}

%%----------------------------------------------------------
\title[Numerical Analysis]{Numerical Analysis}
\subtitle{Chapter Three: Solution of Nonlinear Equations}
\author[zhu]{主讲人~~~~~\textcolor{olive}{朱亚菲}\\
    \quad 幻灯片制作~~\textcolor{olive}{朱亚菲}}
\institute[中国海洋大学]{\small\textcolor{violet}{中国海洋大学~~信息科学与工程学院}}
\date{2013~年~9~月~27~日}
%\titlegraphic{\vspace{-6em}\includegraphics[height=7cm]{ouc}\vspace{-6em}}
\frame{ \titlepage }
%%----------------------------------------------------------
\section*{Contents}
\frame{\frametitle{Contents}\tableofcontents}



%%----------------------------------------------------------
\section{3.5 Computing Roots of Polynomials}

\subsection{3.5.3 Laguerre Iteration}

\begin{frame}
  \frametitle{3.5.3 Laguerre Iteration}
  Laguerre's method is also used for finding the roots of a polynomial $p$.
  \begin{itemize}
  \item Advantages:
  \begin{itemize}
  \item Favorable convergence properties
  \item Rather robust
  \end{itemize}
  \end{itemize}
  The algorithm is iterative and proceeds from one approximate root $z$ to a new one by calculating
  \[ A=-p'(z)/p(z) \]
  \[ B=A^2-p''(z)/p(z) \]
  \[ C=n^{-1}[A\pm \sqrt{(n-1)(nB-A^2)}] \]
  \[z_{new} =z+1/C \]
\end{frame}


\begin{frame}
  \frametitle{3.5.3 Laguerre Iteration}
  \newtheorem*{theorem 10}{THEOREM 10}
  \begin{theorem 10}[Theorem on Radius of Convergences in Laguerre's Method]
  If $p$ is a polynomial of degree $n$, if $z$ is any complex number, and if $C$ is computed as in Laguerre's algorithm, then $p$ has a root in the complex plane within distance $\sqrt n/C$ of $z$.
  \end{theorem 10}
\end{frame}


\begin{frame}
  \frametitle{3.5.3 Laguerre Iteration}
  \lstinputlisting{./Laguerre.m}
\end{frame}


\begin{frame}
  \frametitle{3.5.3 Laguerre Iteration}
  \newtheorem{lemma 1}{LEMMA}
  \begin{lemma 1}[First Lemma on Interval Endpoints]
  Let $v_1$, $v_2$, \ldots , $v_n$ be any real numbers. Put $\alpha=\sum_{i=1}^n v_i$ and $ \beta=\sum_{i=1}^n v_i^2$. Then the numbers $v_i$ lie in the closed interval whose endpoints are
  \begin{displaymath}
  n^{-1}[\alpha \pm \sqrt{(n-1)(n\beta-\alpha^2)}]
  \end{displaymath}
  \end{lemma 1} 
\end{frame}


\begin{frame}
  \frametitle{3.5.3 Laguerre Iteration}
  \begin{proof}
  It suffices to proof that $v_1$ lies in the interval described. Recall the Cauchy-Schwarz Inequality:
  \[ \left( \sum_{i=1}^m x_iy_i \right)^2 \le \left(\sum_{i=1}^m x_i^2\right)\left(\sum_{j=1}^m y_j^2\right) \]
  Applying this, we have
  \begin{displaymath}
  \alpha^2-2\alpha v_1+v_1^2=(\alpha-v_1)^2=(v_2+v_3+\ldots+v_n)^2
  \end{displaymath}
  \begin{displaymath}
  \le(1^2+1^2+\ldots+1^2)(v_2^2+v_3^2+\ldots+v_n^2)=(n-1)(v_2^2+v_3^2+\ldots+v_n^2)
  \end{displaymath}
  \begin{displaymath}
  =(n-1)(\beta-v_1^2)=(n-1)\beta-nv_1^2+v_1^2
  \end{displaymath}
  \end{proof}
\end{frame}


\begin{frame}
  \frametitle{3.5.3 Laguerre Iteration}
  \begin{proof}
  Rearranging this inequality gives us
  \begin{displaymath}
  nv_1^2-2\alpha v_1+\alpha^2-(n-1)\beta \le0
  \end{displaymath}
  This shows that the quadratic function $q(x)=nx^2-2\alpha x+\alpha^2-(n-1)\beta$ has the property $q(v_1)\le0$. For large |x|, obviously $q(x)>0$. hence, $v_1$ lies between the two roots of $q$, and they are the endpoints in Formula $n^{-1}[\alpha \pm \sqrt{(n-1)(n\beta-\alpha^2)}]$
  \end{proof}
\end{frame}


\begin{frame}
  \frametitle{3.5.3 Laguerre Iteration}
  \newtheorem{lemma 2}[lemma 1]{LEMMA}
  \begin{lemma 2}[Second Lemma on Interval Endpoints]
  Let $p$ be a real polynomial of degree $n$, whose roots, $r_1$, $r_2$, \ldots , $r_n$ are real. For any real $x$ different from all the $r_j$, the numbers $(x-r_j)^{-1}$ lie in the interval whose endpoints are
  \begin{displaymath}
  [n p(x)]^{-1}\big \{p'{x}\pm \sqrt{[(n-1)p'(x)]^2-n(n-1)p(x)p''(x)}\big \}
  \end{displaymath}
  \end{lemma 2} 
\end{frame}


\begin{frame}
  \frametitle{3.5.3 Laguerre Iteration}
  \newtheorem*{theorem 11}{THEOREM 11}
  \begin{theorem 11}[Theorem on Monotonic Convergence of Laguerre Methed]
  Let $p$ be a real polynomial whose roots are all real. The sequence produced by the Laguerre algorithm with an arbitrary starting point converges monotonically to a root of $p$.
  \end{theorem 11}
\end{frame}


\subsection{3.5.4 Complex Newton's Method}


%如果你想书签不出现问题,请不要用\XeTeX
                                 %这类复杂的指令,直接写XeTeX吧

\begin{frame}
  \frametitle{3.5.4 Complex Newton's Method}
  For a polynomial having complex coefficients, Newton's method should be programmed in complex arithmetic. \\
  After one root has been found, the deflation process (also programmed in complex arithmetic) should be used. Newton's method can then be applied to the reduced polynomial. This process can be repeated until all roots have been determined.
\end{frame}


\begin{frame}
  \frametitle{3.5.4 Complex Newton's Method}
  Further analysis and experience indicate that in general, the procedure is satisfactory provided that two precautions are taken:
  \begin{enumerate}
  \item The roots should be computed in order of increasing magnitude.
  \item Any root obtained by using Newton's method on a reduced polynomial should be immediately refined by applying Newton's method to the original polynomial with the best estimate of the root as the starting value. Only after this has been done should the next step of deflation be carried out.
  \end{enumerate}
\end{frame}


\section{3.6 Homotopy and Continuation Methods}

\subsection{3.6.1 Basic Concepts}


\begin{frame}
  \frametitle{3.6.1 Basic Concepts}
  \begin{itemize}
  \item In general, a homotopy can be any continuous connection between $f$ and $g$.
  \item An example of a \textbf{homotopy} that connects the two functions $f$ and $g$. 
  \[ h(t,x)=tf(x)+(1-t)g(x) \]
  The parameter $t$ runs over the inteval [0,1].
  \end{itemize}
\end{frame}


\begin{frame}
  \frametitle{3.6.1 Basic Concepts}
  The basic idea of the continuation method is to embed the given problem in a one-parameter family of problems, using a parameter $t$ that   runs over the inteval [0,1]. The original problem is made to correspond to $t=1$, and another problem with known solution is made to correspond to $t=0$. For example, we can define
  \[ h(t,x)=tf(x)+(1-t)[f(x)-f(x_0)]=f(x)+(t-1)f(x_0) \] 
  Here $x_0$ can be any point in $X$, and it is clear that $x_0$ will be a solution of the problem when $t=0$.
\end{frame}


\begin{frame}
  \frametitle{3.6.1 Basic Concepts}
  If the equation $h(t,x)=0$ has a unique root for each $t \in [0,1]$, then that root is a function of $t$, and we can write $x(t)$ as the unique member of $X$ that makes the equation $h(t,x(t))=0$ true. The set
  \[ \{ x(t):0 \le t \le 1 \} \]
  can be interpreted as an arc or curve in $X$, paramatrized by $t$. This arc leads from the known point $x(0)$ to the solution of our problem, $x(1)$. The continuation method attempts to determine this curve by computing points on it, $x(t_0),x(t_1),\ldots,x(t_m)$. 
\end{frame}


\begin{frame}
  \frametitle{3.6.1 Basic Concepts}
  If the function $t\mapsto x(t)$ is differentiable and if $h$ is differentiable, then the Implicit Function Theorem enables us to compute $x'(t)$. By pursuing this idea, we can describe the curve in $\{ x(t):0 \le t \le 1 \}$ by a differential equation. Assuming an arbitrary homotopy, we have
  \begin{displaymath}
  0=h\big( t,x(t) \big)
  \end{displaymath}
\end{frame}


\begin{frame}
  \frametitle{3.6.1 Basic Concepts}
  On differentiating with respect to $t$, we obtain
  \begin{displaymath}
  0=h_t \big( t,x(t) \big) +h_x \big( t,x(t) \big) x'(t)
  \end{displaymath}
  in which subscripts denote partial derivatives. Thus,
  \begin{displaymath}
  x'(t)=-[h_x\big( t,x(t)\big) ]^{-1} h_t \big( t,x(t) \big)
  \end{displaymath}
  This is a differential equation for $x$. It has a known initial value because $x(0)$ is supposedly known. On integrating this differential equation, we shall have the value $x(1)$, which is the solution.
\end{frame}


\begin{frame}
  \frametitle{3.6.1 Basic Concepts}
  \begin{exampleblock}{EXAMPLE 1}
  We let $X=Y=R^2$, and define
  \begin{displaymath}
  f(x) =
  \left[ \begin{array}{ccc}
  \xi_1^2-3\xi_2^2+3\\
  \xi_1\xi_2+6
  \end{array} \right]\qquad
  x=(\xi_1,\xi_2)\in R^2
  \end{displaymath}
  \end{exampleblock}
\end{frame}


\begin{frame}
  \frametitle{3.6.1 Basic Concepts}
  \newtheorem{theorem}{THEOREM}
  \begin{theorem}[Theorem on Continuously Differentiable Solution]
  If $f:R^n \to R^n$ is continuously differential and if $||[f'(x)]^{-1}|| \le M$ on $R^n$, then for any $x_0 \in R^n$ there is a unique curve ${x(t):0\le t \le 1}$ in $R^n$ such that $f(x(t))+(t-1)f(x_0)=0$, with $0\le t \le 1$. The function $t\mapsto x(t)$ is a continuously differentiable solution of the initial-value problem $x'=-[f'(x)]^{-1}f(x_0)$, where $x(0)=x_0$.
  \end{theorem}
\end{frame}


\subsection{3.6.2 Tracing the Path}

\begin{frame}
  \frametitle{3.6.2 Tracing the Path}
  Another way of tracing the path $x(t)$:\\
  We start with the equation $h(t,x)=0$, supposing that $x \in R^n$ and $t \in [0,1]$. A vector $y \in R^{n+1}$ is defined by
  \[ y=(t,\xi_1,\xi_2,\ldots,\xi_n) \]
  where $\xi_1,\xi_2,\ldots,\xi_n$ are the components of $x$. Hence, our equation is simply $h(y)=0$. Each component of $y$, including $t$, is now allowed to be a function of an independent variable $s$, and we write $h(y(s))=0$.
\end{frame}


\begin{frame}
  \frametitle{3.6.2 Tracing the Path}
  Differentiating with respect to $s$, we obtain the basic differential equation
  \[ h'(y(s))y'(s)=0 \]
  The variable $s$ starts at 0, as does $t$. The initial value of $x$ is $x(0)=x_0$. Thus, suitable starting values are available for the differential equation.
\end{frame}


\begin{frame}
  \frametitle{3.6.2 Tracing the Path}
  Since $f$ and $g$ are maps of $R^n$ into $R^n$, $h$ is a map of $R^{n+1}$ into $R^n$. The derivative $h'(y)$ is therefore represented by an $n \times (n+1)$ matrix, $A$. The vector $y(s)$ has $n+1$ components, which we denote by $\eta_1,\eta_2,\ldots,\eta_{n+1}$. By appealing to the lemma below, we can obtain another form for the Equation; namely,
  \[ {n_j}'=(-1)^{j+1}det(A_j) \qquad (1\le j\le n+1) \]
  where $A_j$ is the $n \times n$ matrix that results from $A$ by deleting the $j$th column. 
\end{frame}


\begin{frame}
  \frametitle{3.6.2 Tracing the Path}
  \begin{exampleblock}{EXAMPLE 2}  
  Taking $f$ and $x_0$ as in Example 1, we have 
  \begin{displaymath}
  h(t,x)=
  \left[ \begin{array}{ccc}
  \xi_1 ^2-3\xi_2 ^2+2+t\\
  \xi_1\xi_2-1+7t
  \end{array} \right]
  \end{displaymath}
  \end{exampleblock}
\end{frame}


\begin{frame}
  \frametitle{3.6.2 Tracing the Path}
  \newtheorem{lemma}{LEMMA}
  \begin{lemma}[Lemma on Solution of Homogeneous Equation]
  Let A be an $n\times(n+1)$ matrix. A solution of the homogeneous equation $Ax=0$ is given by $x_j=(-1)^j det(A_j)$, where $A_j$ is $A$ without column $j$.
  \end{lemma}
\end{frame}


\subsection{3.6.3 Relation to Newton's Method}

\begin{frame}
  \frametitle{3.6.3 Relation to Newton's Method}
  The connection between the homotopy methods and Newton's method is deeper than may be seen at first glance. Let us start with the homotopy
  \[ h(t,x)=f(x)-e^{-t}f(x_0) \]
  In this equation, $t$ will run from 0 to $\infty$. We seek a curve or path, $x=x(t)$, on which
  \[ 0=h \big( t,x(t) \big) =f \big( x(t) \big) -e^{-t}f(x_0) \]
\end{frame}


\begin{frame}
  \frametitle{3.6.3 Relation to Newton's Method}
  As usual, differentiation with respect to $t$ will lead to a differential equation describing the path:
  \begin{displaymath}
  0=f'\big(x(t)\big)x'(t)+e^{-t}f(x_0)=f'\big(x(t)\big)x'(t)+f\big(x(t)\big)
  \end{displaymath}
  \begin{displaymath}
  x'(t)=-[f'\big(x(t)\big)]^{-1}f\big(x(t)\big)
  \end{displaymath}
  In this differential equation is integrated with Euler's method, with step size 1, the result is the formula
  \[ x_{n+1}=x_{n}-[f'(x_n)]^{-1}f(x_n) \]
\end{frame}


\subsection{3.6.4 Linear Programming}

\begin{frame}
  \frametitle{3.6.4 Linear Programming}
  Consider the standard linear programming problem
  \begin{displaymath}
  \left\{ \begin{array}{ll}
  maximize \ c^Tx\\
  subject \ to \ Ax=b \ and \ x\ge0
  \end{array} \right.
  \end{displaymath}
  Here, $c\in R^n,x\in R^n,b\in R^m$, and A is an $m\times n$ matrix.We start with a $\textbf{feasible point}$---that is, a point $x^0$ that satisfies the constraints. The $\textbf{feasible set}$ is
  \[ \digamma=\{x\in R^n:Ax=b \ and \ x \ge0\} \]
\end{frame}


\begin{frame}
  \frametitle{3.6.4 Linear Programming}
  We shall try to find a curve $t\mapsto x(t)$ in the feasible set, starting at $x^0$ and leading to a solution of the extremal problem.\\
  Our requirements are
  \begin{enumerate}
  \item $x(t)\ge0$ for $t\ge0$
  \item $Ax(t)=b$ for $t\ge0$
  \item $c^Tx(t)$ is increasing for $t\ge0$.
  \end{enumerate}
\end{frame}


\begin{frame}
  \frametitle{3.6.4 Linear Programming}
  The curve will be defined by an initial-value problem:
  \begin{displaymath}
  \left\{ \begin{array}{ll}
  x'=f(x)\\
  x(0)=x^{(0)}
  \end{array} \right.
  \end{displaymath}
  The task facing us is to determine a suitable $f$. 
\end{frame}


\begin{frame}
  \frametitle{3.6.4 Linear Programming}
  To satisfy Condition 1, we shall arrange that whenever a component $x_i$ approaches 0, its velocity $x_i'(t)$ will also approach 0.\\ 
  This can be accomplished by putting
  \begin{displaymath}
  D(x)=
  \left[ \begin{array}{cccc}
  x_1 &     &        & 0 \\
      & x_2 &        &   \\
      &     & \ddots &   \\
   0  &     &        & x_n
  \end{array} \right]
  \end{displaymath}
  and assuming that for some bounded function G,
  \[ f(x)=D(x)G(x) \]
  Then we have $x_i'=x_iG_i(x)$ \\
  and clearly $x_i'\to 0$ if $x_i\to0$.
\end{frame}


\begin{frame}
  \frametitle{3.6.4 Linear Programming}
  To satisfy Requirement 2, it suffices to require $Ax'=0$. \\
  Since $x'=f=DG$, we must require $ADG=0$. \\
  This is most conveniently arranged by letting $G=PH$ where $H$ is any funtion and $P$ is the orthogonal projection onto the null space of $AD$.
\end{frame}


\begin{frame}
  \frametitle{3.6.4 Linear Programming}
  To satisfy Property 3, we should select $H$ so that $c^Tx(t)$ is increasing. Thus, we want
  \[ 0<(c^Tx(t))'=c^Tx'=c^Tf(x)=c^TDG=c^TDPH \]
  A convenient choice for $H$ is $Dc$ because then we have, with $v=Dc$,
  \[ c^TDPH=c^TDPDc=v^TPv=\langle v,Pv\rangle=\langle v-Pv+Pv,Pv\rangle=\langle Pv,Pv\rangle \ge0 \]
  Notice that $v-Pv$ is orthogonal to the range of $P$, and $\langle v-Pv,Pv\rangle=0$.
\end{frame}


\begin{frame}
  \frametitle{3.6.4 Linear Programming}
  The final version of our initial-value problem is
  \[ x'=D(x)P(x)D(x)c \quad x(0)=x^{(0)} \]
\end{frame}


\end{document}
