\documentclass[notheorems,mathserif,table,compress]{beamer}  %dvipdfm选项是关键,否则编译统统通不过
%%------------------------常用宏包------------------------
%%注意, beamer 会默认使用下列宏包: amsthm, graphicx, hyperref, color, xcolor, 等等
\usepackage{fontspec,xunicode,xltxtra}  % for XeTeX
\usepackage{verbatim}
\usepackage{mathabx}
\usepackage{amsfonts,amssymb}
\usepackage{iplouclistings}
%%------------------------ThemeColorFont------------------------
%% Presentation Themes
% \usetheme[<options>]{<name list>}
\usetheme{Madrid}
%% Inner Themes双精度计算
% \useinnertheme[<options>]{<name>}
%% Outer Themes
% \useoutertheme[<options>]{<name>}
\useoutertheme{miniframes} 
%% Color Themes 
%\usecolortheme[<options>]{<name list>}
%% Font Themes
\usefonttheme{serif}
\setbeamertemplate{background canvas}[vertical shading][bottom=white,top=structure.fg!7] %%背景色, 上25%的蓝, 过渡到下白.
\setbeamertemplate{theorems}[numbered]
\setbeamertemplate{navigation symbols}{}   %% 去掉页面下方默认的导航条.
\usepackage{zhfontcfg}
%\setsansfont[Mapping=tex-text]{文泉驿正黑}  %% 需要fontspec宏包
     %如果装了Adobe Acrobat,可在font.conf中配置Adobe字体的路径以使用其中文字体
     %也可直接使用系统中的中文字体如SimSun,SimHei,微软雅黑 等
     %原来beamer用的字体是sans family;注意Mapping的大小写,不能写错
     %设置字体时也可以直接用字体名,以下三种方式等同:
     %\setromanfont[BoldFont={黑体}]{宋体}
     %\setromanfont[BoldFont={SimHei}]{SimSun}
     %\setromanfont[BoldFont={"[simhei.ttf]"}]{"[simsun.ttc]"}
%%------------------------MISC------------------------
\graphicspath{{figures/}}         %% 图片路径. 本文的图片都放在这个文件夹里了.
%%------------------------正文------------------------
\begin{document}
\XeTeXlinebreaklocale "zh"         % 表示用中文的断行
\XeTeXlinebreakskip = 0pt plus 1pt % 多一点调整的空间
%%----------------------------------------------------------
%% This is only inserted into the PDF information catalog. Can be left
%% out.
%%%
%% Delete this, if you do not want the table of contents to pop up at
%% the beginning of each subsection:
\AtBeginSection[]{                              % 在每个Section前都会加入的Frame
  \frame<handout:0>{
    \frametitle{Contents}\small
    \tableofcontents[current,currentsubsection]
  }
}

\AtBeginSubsection[]                            % 在每个子段落之前
{
  \frame<handout:0>                             % handout:0 表示只在手稿中出现
  {
    \frametitle{Contents}\small
    \tableofcontents[current,currentsubsection] % 显示在目录中加亮的当前章节
  }
}

%%----------------------------------------------------------
\title[Numerical Analysis]{Numerical Analysis}
\subtitle{Mathematics of Scientific Computing}
\author[qiu]{主讲人~~~~~\textcolor{olive}{邱欣欣}\\
    \quad 幻灯片制作~~\textcolor{olive}{邱欣欣}}
\institute[中国海洋大学]{\small\textcolor{violet}{中国海洋大学~~信息科学与工程学院}}
\date{2013~年~9~月~27~日}
%\titlegraphic{\vspace{-6em}\includegraphics[height=7cm]{ouc}\vspace{-6em}}
\frame{ \titlepage }
%%----------------------------------------------------------
\section*{Contents}
\frame{\frametitle{Solution of Nonlinear Equations}\tableofcontents}
%%----------------------------------------------------------
\section{Fixed Points and Functional Iteration}

%4
\begin{frame}
\frametitle{Fixed Points and Functional Iteration}
\begin{itemize}
\item A sequence of points is computed by a formula of the form
\begin{equation}
x_{n+1}=F(x_{n}) \quad(n\geq0) 
\end{equation}

The algorithm defined by such an equation is called \textsf{functional iteration}.\\
In Newton's method, the function $F$ is given by
\begin{displaymath}
F(x)=x-\frac{f(x)}{f'(x)}
\end{displaymath}

whereas in Steffensen's method, we have
\begin{displaymath}
F(x)=x-\frac{[f(x)]^2}{f(x+f(x))-f(x)}
\end{displaymath}

\end{itemize}
\end{frame}

%5
\begin{frame}
\frametitle{Fixed Points and Functional Iteration}
\begin{itemize}
\item Suppose that
\begin{displaymath}
\lim_{n \to \infty}x_n=s
\end{displaymath}
 If $F$ is continuous, then 
\begin{displaymath}
F(s)=F(\lim_{n \to \infty}x_n)=\lim_{n\to\infty}F(x_n)=\lim_{n\to\infty}x_{n+1}=s
\end{displaymath}

Thus, $F(s)=s$, and we call $s$ a fixed point of the function $F$.
\end{itemize}
\begin{itemize}
\item The theorem to be proved concerns \textsf{contractive mapping}. A mapping(or function) $F$ is said to be \textsf{contractive} if there exists a number $\lambda$ less than 1 such that
\begin{equation}
|F(x)-F(y)|\leq \lambda|x-y|
\end{equation}

for all points $x$ and $y$ in the domain of $F$.
\end{itemize}
\end{frame}

%6
\begin{frame}
\frametitle{Contractive Mapping Theorem}
\theoremstyle{plain}
\newtheorem{theorem}{THEOREM}
\begin{theorem}[Contractive Mapping Theorem]
Let C be a closed subset of the real line. If F is a contractive mapping of C into C, then F has a unique fixed point. Moreover, this fixed point is the limit of every sequence obtained from Equation(1) with a starting point $x_0\in C$.
\end{theorem}

\begin{proof}
We use the contractive property(2) together with Equation(1) to write
\begin{displaymath}
|x_n-x_{n-1}|=|F(x_{n-1})-F(x_{n-2})|\leq\lambda|x_{n-1}-x_{n-2}|
\end{displaymath}

This argument can be repeated to get
\begin{displaymath}
|x_n-x_{n-1}|\leq\lambda|x_{n-1}-x_{n-2}|\leq{\lambda}^2|x_{n-2}-x_{n-3}|\leq\ldots\leq{\lambda}^{n-1}|x_1-x_0|
\end{displaymath}

\end{proof}
\end{frame}

%7
\begin{frame}
\frametitle{Contractive Mapping Theorem}
\begin{proof}
Since $x_n$ can be written in the form
\begin{displaymath}
x_n=x_0+(x_1-x_0)+(x_2-x_1)+\ldots+(x_n-x_{n-1})
\end{displaymath}

We see that the sequence $[x_n]$ converges if and only if the series
\begin{displaymath}
\sum_{n=1}^{\infty}(x_n-x_{n-1})
\end{displaymath}

converges. It suffices to prove that the series
\begin{displaymath}
\sum_{n=1}^{\infty}|x_n-x_{n-1}|
\end{displaymath}
converges. 
\end{proof}
\end{frame}

%8
\begin{frame}
\frametitle{Contractive Mapping Theorem}
\begin{proof}
This is easy because we can use the comparision test and the previous work
\begin{displaymath}
\sum_{n=1}^{\infty}|x_n-x_{n-1}|\leq\sum_{n=1}^{\infty}\lambda ^{n-1}|x_1-x_0|=\frac{1}{1-\lambda}|x_1-x_0|
\end{displaymath}

Let $s=\lim_{n\to \infty}x_n$. Then $F(s)=s$, as noted previously. \\
As for the unicity of the fixed point, if $x$ and $y$ are fixed points, then 
\begin{displaymath}
|x-y|=|F(x)-F(y)|\leq\lambda|x-y|
\end{displaymath}

Since $\lambda<1,|x-y|=0$. Finally, note that the point $s$ that is obtained belongs to $C$ since $s$ is the limit of a sequence lying in $C$.
\end{proof}
\end{frame}

%9
\begin{frame}
\frametitle{Fixed Points and Functional Iteration}
\begin{description}
\item[EXAMPLE] Use the Contractive Mapping Theorem to compute a fixed point of the function
\begin{displaymath}
F(x)=4+\frac{1}{3}\sin x
\end{displaymath}

\end{description}
\begin{description}
\item[Solution] By the Mean-Value Theorem, we have
\begin{displaymath}
|F(x)-F(y)|=\frac{1}{3}|\sin 2x-\sin 2y|=\frac{2}{3}|\cos 2\zeta||x-y|\leq\frac{2}{3}|x-y|
\end{displaymath}

for some $\zeta$ between $x$ and $y$. This shows that $F$ is contractive, with $\lambda=2/3$. By Theorem 1, $F$ has a fixed point. A computer program to compute this fixed point can be based on the following algorithm.
\end{description}
\end{frame}

%10
\begin{frame}
\frametitle{Contractive Mapping Theorem}
\lstinputlisting{./contractive.m}
\end{frame}

%11
\begin{frame}
\frametitle{Error Analysis}
\begin{itemize}
\item We suppose $F$ has a fixed point, $s$, and that a sequence $[x_n]$ has been defined by the formula $x_{n+1}=F(x_n)$. Let 
\begin{displaymath}
e_n=x_n-s
\end{displaymath}

If $F'$ exists and is continuous, then by the Mean-Value Theorem,
\begin{displaymath}
x_{n+1}-s=F(x_n)-F(s)=F'({\zeta}_n)(x_n-s)\quad or\quad e_{n+1}=F'({\zeta}_n)e_n
\end{displaymath}

where ${\zeta}_n$ is a point between $x_n$ and $s$. The condition $|F'(x)|<1$ for all $x$ ensures that errors decrease in magnitude. If $e_n$ is small, then ${\zeta}_n$ is near $s$, and $F'({\zeta}_n)\approx F'(s)$. One would expect rapid convergence if $F'(s)$ is small.
\end{itemize}
\end{frame}

\section{Computing Roots of Polynomials}

%13
\begin{frame}
\frametitle{Computing Roots of Polynomials}
\begin{itemize}
\item We write a polynomial in the form 
\begin{equation}
p(z)=a_n z^n+a_{n-1}z^{n-1}+\ldots+a_2 z^2+a_1z+a_0
\end{equation}

in which the coefficients $a_k$ and the variable $z$ may be complex numbers. If $a_n\neq0$, then $p$ has \textsf{degree} n. We are interested in finding the roots of $p$.
\end{itemize}
\begin{theorem}[Fundamental Theorem of Algebra]
Every nonconstant polynomial has at least one root in the complex field.
\end{theorem}

\end{frame}

%14
\begin{frame}
\frametitle{Fundamental Theorem of Algebra}
\begin{proof}
We want to show that $p(z_0)=0$ for some $z_0\in \mathbb{C}$. Since $p$ is not constant, $|p(z)|\to\infty$ when $|z|\to\infty$. Let $D$ be a disk centered at $0$ outside of which $|p(z)|\geq|p(0)|$. Let $z_0$ be a point where $inf_{z\in D}|P(z)|$ is attained. Since $0\in D$, $|P(z_0)|\leq|p(0)|$. Thus, $|p(z_0)|\leq|p(z)|$ for all $z\in \mathbb{C}$. Put $q(z)=p(z+z_0)$. We want to prove that $q(0)=0$, so that $p(z_0)=0$. Write $q(z)=c_0+c_j z^j+\ldots+c_n z^n=c_0+c_j z^j+z^{j+1}r(z)$, in which $c_j\neq0$ and $r$ is a polynomial. Now we want to prove that $c_0=0$.
\end{proof}
\end{frame}

%15
\begin{frame}
\frametitle{Fundamental Theorem of Algebra}
\begin{proof}
Suppose $c_0\neq0$. Let $w$ be any complex number such that $c_j w^j=-c_0$. Define $N=sup_{0<\varepsilon<1}|r(\varepsilon w)|$. Select $\varepsilon$ in $(0,1)$ so small that $\varepsilon|w|^{j+1}N<|c_0|$. Then we obtain a contradiction as follows:
\begin{equation*}
\begin{split}
|q(\varepsilon w)| & \leq|c_0+c_j \varepsilon^j w^j|+\varepsilon^{j+1}|w^{j+1}||r(\varepsilon w)| \\
                   & =|c_0-c_0\varepsilon^j|+\varepsilon^j\varepsilon|w|^{j+1}N\\
                   & <|c_0|(1-\varepsilon^j)+\varepsilon^j|c_0|=|c_0|=|q(0)|\\
                   & =|p(z_0)|\leq|p(z_0+\varepsilon w)|=|q(\varepsilon w)|
\end{split}
\end{equation*}

\end{proof}
\end{frame}

%16
\begin{frame}
\frametitle{Computing Roots of Polynomials}
\begin{itemize}
\item If the polynomial $p$, having degree $n$ at least 1, is divided by a linear factor $z-c$, the result ia a quotient $q$ and a remainder $r$. The latter is a complex number, and the former is a polynomial of degree $n-1$. We can represent the process by the equation
\begin{displaymath}
p(z)=(z-c)q(z)+r 
\end{displaymath}

\end{itemize}
\begin{itemize}
\item From this we see (by letting $z=c$) that $p(c)=r$. This fact is known as the \textsf{Remainder Theorem}. 
\end{itemize}
\begin{itemize}
\item If $c$ is a root of $p$, then $r=0$. This implication is known as the \textsf{Factor Theorem}. 
\end{itemize}
\end{frame}

%17
\begin{frame}
\frametitle{Computing Roots of Polynomials}
\begin{itemize}
\item Let us write $p(z)=(z-r_1)q_1(z)$, where $r_1$ is any root of $p$. The equation can be
\begin{displaymath}
p(z)=(z-r_1)(z-r_2)\ldots(z-r_n)q_n
\end{displaymath}

This proves that a polynomial of degree $n$ has a factorization into a product of linear factors, each corresponding to a root of $p$. It is clear that $p$ can have no other roots. Since some of the roots $r_k$ may be equal to each other, we see that a polynomial of degree $n$ can have at most $n$ roots.
\end{itemize}
\end{frame}

%18
\begin{frame}
\frametitle{Computing Roots of Polynomials}
\begin{theorem}[Theorem on Complex Roots of Polynomials]
A polynomial of degree $n$ has exactly $n$ roots in the complex plane, it being agreed that each root shall be counted a number of times equal to its multiplicity.
\end{theorem}

\begin{theorem}[Localization Theorem]
All roots of the polynomial in Equation(3) lie in the open disk whose center is at the origin of the complex plane and whose radius is
\begin{displaymath}
\rho=1+|a_n|^{-1}\max_{0\leq k<n}|a_k|
\end{displaymath}

\end{theorem}
\end{frame}

%19
\begin{frame}
\frametitle{Computing Roots of Polynomials}
\begin{proof}
Put $c=\max_{0\leq k<n}|a_k|$ so that $c|a_n|^{-1}=\rho-1$. If $c=0$, our result is trivially true. Hence, assume $c>0$. Then $\rho>1$. If $|z|\geq \rho$, then (because $\rho>1$)
\begin{equation*}
\begin{split}
|p(z)| & \geq |a_n z^n|-|a_{n-1} z^{n-1}+\ldots+a_0| \\
       & \geq|a_n z^n|-c\sum_{k=0}^{n-1}|z|^k\\
       & >|a_n z^n|-c|z|^n(|z|-1)^{-1}\\
       & =|a_n z^n|\{1-c|a_n|^{-1}(|z|-1)^{-1}\}\\
       & \geq|a_n z^n|\{1-c|a_n|^{-1}(\rho-1)^{-1}\}=0
\end{split}
\end{equation*}

\end{proof}
\end{frame}

%20
\begin{frame}
\frametitle{Computing Roots of Polynomials}
Take the polynomial $p$ of Equation(3), and consider the function $s(z)=z^n p(1/z)$. Then
\begin{equation*}
\begin{split}
s(z) & =z^n[a_n\left(\frac{1}{z}\right)^n+a_{n-1}\left(\frac{1}{z}\right)^{n-1}+\ldots+a_0]\\
     & =a_n+a_{n-1}z+a_{n-2}z^2+\ldots+a_0 z^n
\end{split}
\end{equation*}

This shows that $s$ is a polynomial of degree at most $n$. For a nonzero complex number $z_0$, the condition $p(z_0)=0$ is equivalent to the condition $s(1/z_0)=0$.
\begin{theorem}[Theorem on Localization of Roots]
If all the roots of s are in the disk $\{z:|z|\leq \rho\}$, then all the nonzero roots of p are outside the disk $\{z:|z|<\rho^{-1}\}$.
\end{theorem}
\end{frame}

\subsection{Horner's Algorithm}

%22
\begin{frame}
\frametitle{Horner's Algorithm}
\begin{itemize}
\item If a polynomial $p$ and a complex number $z_0$ are given, Horner's algorithm will produce the number $p(z_0)$ and the polynomial 
\begin{displaymath}
q(z)=\frac{p(z)-p(0)}{z-z_0}
\end{displaymath}

From this equation we have
\begin{equation}
p(z)=(z-z_0)q(z)+p(z_0)
\end{equation}

Let the unknown polynomial $q$ be represented by
\begin{displaymath}
q(z)=b_0+b_1 z+\ldots+b_{n-1} z^{n-1}
\end{displaymath}

Then  this form for $q(z)$ and the analogous for $p(z)$ are substituted into Equation(4).
\end{itemize}
\end{frame}

%23
\begin{frame}
\frametitle{Horner's Algorithm}
\begin{itemize}
\item The coefficients of like powers of $z$ on the two sides of the equation can be set equal to each other.
These equations arise from doing so:
\begin{equation*}
\begin{split}
b_{n-1} & = a_n\\
b_{n-2} & = a_{n-1}+z_0 b_{n-1}\\
\vdots \\
    b_0 & = a_1+z_0 b_1\\
 p(z_0) & = a_0+z_0 b_0
\end{split}
\end{equation*}

\end{itemize}
\begin{itemize}
\item The calculation can be carried out in the following arrangement
\begin{tabular}{r|c c c c c}
      & $a_n$ & $a_{n-1}$ & $a_{n-2}$ & $\ldots$ & $a_0$\\
$z_0$ &       & $z_0 b_{n-1}$ & $z_0 b_{n-2}$ & $\ldots$ & $z_0 b_0$\\
\hline
      & $b_{n-1}$ & $b_{n-2}$ & $b_{n-3}$ & $\ldots$ & $b_{-1}$
\end{tabular}

\end{itemize}
\end{frame}

%24
\begin{frame}
\frametitle{Horner's Algorithm}
\lstinputlisting{./Horner1.m}
\end{frame}

%25
\begin{frame}
\frametitle{Horner's Algorithm}
\begin{itemize}
\item A application of Horner's algorithm is in finding the Taylor expansion of a polynomial about any point. Let $p(z)$ be as in Equation(3), and suppose that we desire the cofficient $c_k$ in the equation
\begin{equation*}
\begin{split}
p(z) & =a_n z^n+a_{n-1} z^{n-1}+\ldots+a_0\\
     & =c_n(z-z_0)^n+c_{n-1}(z-z_0)^{n-1}+\ldots+c_0
\end{split}
\end{equation*}
 
Notice that $p(z_0)=c_0$. The algorithm yields the polynomial
\begin{displaymath}
q(z)=\frac{p(z)-p(z_0)}{z-z_0}=c_n(z-z_0)^{n-1}+c_{n-1}(z-z_0)^{n-2}+\ldots+c_1
\end{displaymath}

This shows $c_1$ can be obtained by applying Horner's algorithm to the polynomial $q$ with point $z_0$ because $c_1=q(z_0)$. This process is repeated until all coefficients $c_k$ are found.
\end{itemize}
\end{frame}

%26 
\begin{frame}
\frametitle{Horner's Algorithm}
\begin{itemize}
\item We call the algorithm described the \textsf{complete Horner's algorithm}.
\lstinputlisting{./Horner2.m}
\end{itemize}
\end{frame}

%27
\begin{frame}
\frametitle{Horner's Algorithm}
\begin{itemize}
\item We will show how Newton's iteration can be carried out on a polynomial. Here is a code that produces $\alpha=p(z_0)$ and $\beta=p'(z_0)$.
\lstinputlisting{./Horner3.m}
\end{itemize}
\end{frame}

%28
\begin{frame}
\frametitle{Horner's Algorithm}
\begin{itemize}
\item Then a code for taking M steps in Newton's method on the given polynomial, starting at $z_0$, would look like this: 
\lstinputlisting{./NewtonH.m}
\end{itemize}
\end{frame}

%29
\begin{frame}
\frametitle{Theorem on Horner's Method}
\begin{theorem}[Theorem on Horner's Method]
Let $p(x)=a_n x^n+\ldots+a_1 x+a_0$. Define pairs $(\alpha_j,\beta_j)$ for $j=n,n-1,\ldots,0$ by the algorithm
\begin{displaymath}
\left\{\begin{array}{ll}
(\alpha_n,\beta_n)=(a_n,0)\\
(\alpha_j,\beta_j)=(a_j+x\alpha_{j+1},\alpha_{j+1}+x\beta_{j+1}) & (n-1\geq j\geq 0)
\end{array} \right.
\end{displaymath}

Then $\alpha_0=p(x)$ and $\beta_0=p'(x)$. 
\end{theorem}
\end{frame}

%30
\begin{frame}
\frametitle{Theorem on Successive Newton Iterates}
\begin{theorem}[Theorem on Successive Newton Iterates]
Let $x_k$ and $x_{k+1}$ be two successive iterates when Newton's method is applied to a polynomial $p$ of degree $n$. Then there is a root of $p$ within distance $n|x_k-x_{k+1}|$ of $x_k$ in the complex plane.
\end{theorem}
\begin{proof}
Let $r_1,r_2,\ldots,r_n$ be the roots of $p$. Then $p(z)=c\prod_{j=1}^n (z-r_j)$. The correction term in the Newton iteration is $-p(z)/p'(z)$. The derivative of $p$ is
\begin{displaymath}
p'(z)=c\sum_{k=1}^n \prod_{\substack{i=1 \\ i\neq k}}^n(z-r_i)=\sum_{k=1}^n p(z)/(z-r_k)=p(z)\sum_{k=1}^n (z-r_k)^{-1}
\end{displaymath}

\end{proof}
\end{frame}

%31
\begin{frame}
\frametitle{Theorem on Successive Newton Iterates}
\begin{proof}
For any $z$ (playing the role of $x_k$) there is an index $j$ for which $|z-r_j|\leq n|p(z)/p'(z)|$.\\ If no index $j$ satisfies the desired inequality, then for all $j$, $|z-r_j|> n|p(z)/p'(z)|$. From this it would follow that 
\begin{displaymath}
|z-r_j|^{-1}<\frac{1}{n}|p'(z)/p(z)|=\frac{1}{n}|\sum_{k=1}^n (z-r_k)^{-1}|\leq\frac{1}{n}\sum_{k=1}^n|(z-r_k)^{-1}|
\end{displaymath}

But this is not possible because the average of $n$ numbers cannot be greater than each of them.
\end{proof}
\end{frame}

\subsection{Bairstow's Method}

%33
\begin{frame}
\frametitle{Theorem on Real Quadratic Factor}
\begin{theorem}[Theorem on Real Quadratic Factor]
If $p$ is a polynomial whose coefficients are all real, and if $w$ is a nonreal root of $p$, then $\overline{w}$ is also a root, and $(z-w)(z-\overline w)$ is a real quadratic factor of $p$.
\end{theorem}

\begin{proof}
Let $p(z)=a_n z^n+a_{n-1} z^{n-1}+\ldots+a_1 z+a_0$, with all $a_k$ being real. Since $w$ is a root of $p$, we have
\begin{displaymath}
0=a_n w^n+a_{n-1} w^{n-1}+\ldots+a_1 w+a_0
\end{displaymath}

Take the conjugate of both sides, the result is
\begin{displaymath}
0=a_n {\overline w}^n+a_{n-1} {\overline w}^{n-1}+\ldots+a_1 \overline w+a_0
\end{displaymath}

\end{proof}
\end{frame}

%34
\begin{frame}
\frametitle{Theorem on Quotient and Remainder}
\begin{theorem}[Theorem on Quotient and Remainder]
If the polynomial $p(z)=a_n z^n+a_{n-1} z^{n-1}+\ldots+a_1 z+a_0$ is divided by the quadratic polynomial $z^2-uz-v$, then the quotient and remainder
\begin{equation*}
\begin{split}
q(z) & =b_n z^{n-2}+b_{n-1} z^{n-3}+\ldots+b_3 z+b_2\\
r(z) & =b_1(z-u)+b_0
\end{split}
\end{equation*}

can be computed recursively by setting $b_{n+1}=b_{n+2}=0$ and then using
\begin{displaymath}
b_k=a_k+ub_{k+1}+vb_{k+2} \qquad (n\geq k\geq 0)
\end{displaymath}

\end{theorem}
\end{frame}

%35
\begin{frame}
\frametitle{Theorem on Quotient and Remainder}
\begin{proof}
The relationship between $p,q$, and $r$ is expressed by 
\begin{displaymath}
p(z)=q(z)(z^2-uz-v)+r(z)
\end{displaymath}

In detail, this equation reads:
\begin{displaymath}
\sum_{k=0}^n a_kz^k=(\sum_{k=2}^n b_kz^{k-2})(z^2-uz-v)+b_1(z-u)+b_0
\end{displaymath}

If we equate the coefficients of $z^k$ on the two sides of this equation,
\begin{equation*}
\begin{split}
a_k & =b_k- ub_{k+1}-vb_{k+2}  \qquad (0\leq k\leq n-2)\\
a_{n-1} & =b_{n-1}-ub_n\\
a_n &=b_n
\end{split}
\end{equation*}

\end{proof}
\end{frame}

%36
\begin{frame}
\frametitle{Bairstow's Method}
\begin{proof}
In the division process above, $b_0$ and $b_1$ are functions of $u$ and $v$, and we wirte $b_0=b_0(u,v)$ and $b_1=b_1(u,v)$. In order that $q$ be a factor of $p$, the remainder $r$ should vanish; this leads to the two equations
\begin{equation*}
\begin{split}
b_0(u,v)&=0\\
b_1(u,v)&=0
\end{split}
\end{equation*}

This pair of simultaneous nonlinear equation is solved by Newton's method. We require the paitial derivatives
\begin{displaymath}
c_k=\frac{\partial b_k}{\partial u}\qquad d_k=\frac{\partial b_{k-1}}{\partial v}\qquad (0\leq k\leq n)
\end{displaymath}

\end{proof}
\end{frame}

%37
\begin{frame}
\frametitle{Bairstow's Method}
\begin{proof}
These are obtained by differentiating the recurrence relation already established for $b_k$ in Theorem 9.
\begin{equation*}
\begin{split}
c_k&=b_{k=1}+uc_{k+1}+vc_{k+2} \qquad(c_{n+1}=c_n=0)\\
d_k&=b_{k=1}+ud_{k+1}+vd_{k+2} \qquad(d_{n+1}=d_n=0)
\end{split}
\end{equation*}

Since these recurrence relation generate the same two sequences, we need the first. Starting values are assigned to $u$ and $v$. We seek corrections, donoted by $\delta u$ and $\delta v$, so that the equations 
\begin{displaymath}
b_0(u+\delta u,v+\delta v)=b_1(u+\delta u,v+\delta v)=0
\end{displaymath}

are sure.
\end{proof}
\end{frame}

%38
\begin{frame}
\frametitle{Bairstow's Method}
\begin{proof}
We linearize these equations by writing
\begin{equation*}
\begin{split}
b_0(u,v)+\frac{\partial b_0}{\partial u}\delta u +\frac{\partial b_0}{\partial u}\delta v&=0\\
b_1(u,v)+\frac{\partial b_1}{\partial u}\delta u +\frac{\partial b_1}{\partial u}\delta v&=0
\end{split}
\end{equation*}

In view of the preceding remarks, this system becomes
\begin{displaymath}
\begin{bmatrix}
c_0 & c_1 \\
c_1 & c_2 
\end{bmatrix}
\begin{bmatrix}
\delta u\\
\delta v
\end{bmatrix}
=-\begin{bmatrix}
b_0(u,v) \\
b_1(u,v)
\end{bmatrix}
\end{displaymath}

\end{proof}
\end{frame}

%39
\begin{frame}
\frametitle{Bairstow's Method}
\begin{proof}
The solution of this system follows:
\begin{equation*}
\begin{split}
\delta u&=(c_1 b_1-c_2 b_0)/J\\
\delta v&=(c_1 b_0-c_0 b_1)/J\\
J&=c_0 c_2-c_1^2
\end{split}
\end{equation*}

Notice that $J$ is the Jacobian determinant for the pair of nonlinear function $b_0(u,v)$ and $b_1(u,v)$.
\end{proof}
\end{frame}

%40
\begin{frame}
\frametitle{Bairstow's Method}
\lstinputlisting{./Bairstow.m}
\end{frame}

%41
\begin{frame}
\frametitle{Theorem on the Jacobian in Bairstow's Method}
\begin{theorem}[Theorem on the Jacobian in Bairstow's Method]
Let $(u_0,v_0)$ be a point such that the roots of $z^2-u_0z-v_0$ are simple roots of $p$. The Jacobian in Bairstow's method is not 0 at $(u_0,v_0)$. 
\end{theorem}
\end{frame}

\end{document}
