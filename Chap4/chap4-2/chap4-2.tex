\documentclass[notheorems,mathserif,table,compress]{beamer}  %dvipdfm选项是关键,否则编译统统通不过
%%------------------------常用宏包------------------------
%%注意, beamer 会默认使用下列宏包: amsthm, graphicx, hyperref, color, xcolor, 等等
\usepackage{fontspec,xunicode,xltxtra}
\usepackage{amsfonts,amssymb}  % for XeTeX
%%------------------------ThemeColorFont------------------------
%% Presentation Themes
% \usetheme[<options>]{<name list>}
\usetheme{Madrid}
%% Inner Themes
% \useinnertheme[<options>]{<name>}
%% Outer Themes
% \useoutertheme[<options>]{<name>}
\useoutertheme{miniframes} 
%% Color Themes 
% \usecolortheme[<options>]{<name list>}
%% Font Themes
% \usefonttheme[<options>]{<name>}
\setbeamertemplate{background canvas}[vertical shading][bottom=white,top=structure.fg!7] %%背景色, 上25%的蓝, 过渡到下白.
\setbeamertemplate{theorems}[numbered]
\setbeamertemplate{navigation symbols}{}   %% 去掉页面下方默认的导航条.
\usepackage{zhfontcfg}
\usepackage{iplouclistings}
%\setsansfont[Mapping=tex-text]{文泉驿正黑}  %% 需要fontspec宏包
     %如果装了Adobe Acrobat,可在font.conf中配置Adobe字体的路径以使用其中文字体
     %也可直接使用系统中的中文字体如SimSun,SimHei,微软雅黑 等
     %原来beamer用的字体是sans family;注意Mapping的大小写,不能写错
     %设置字体时也可以直接用字体名,以下三种方式等同:
     %\setromanfont[BoldFont={黑体}]{宋体}
     %\setromanfont[BoldFont={SimHei}]{SimSun}
     %\setromanfont[BoldFont={"[simhei.ttf]"}]{"[simsun.ttc]"}
%%------------------------MISC------------------------
\graphicspath{{figures/}}         %% 图片路径. 本文的图片都放在这个文件夹里了.
%%------------------------正文------------------------
\begin{document}
\XeTeXlinebreaklocale "zh"         % 表示用中文的断行
\XeTeXlinebreakskip = 0pt plus 1pt % 多一点调整的空间
%%----------------------------------------------------------
%% This is only inserted into the PDF information catalog. Can be left
%% out.
%%%
%% Delete this, if you do not want the table of contents to pop up at
%% the beginning of each subsection:
\AtBeginSection[]{                              % 在每个Section前都会加入的Frame
  \frame<handout:0>{
    \frametitle{内容提要}\small
    \tableofcontents[current,currentsubsection]
  }
}
\AtBeginSubsection[]                            % 在每个子段落之前
{
  \frame<handout:0>                             % handout:0 表示只在手稿中出现
  {
    \frametitle{下一节内容}\small
    \tableofcontents[current,currentsubsection] % 显示在目录中加亮的当前章节
  }
}
%%----------------------------------------------------------
\title[Numerical Analysis]{Numerical Analysis}
\subtitle{Mathematics of Scientific Computing}
\author[sun]{主讲人~~~~~\textcolor{olive}{孙晓庆}\\
    \quad 幻灯片制作~~\textcolor{olive}{孙晓庆}}
\institute[中国海洋大学]{\small\textcolor{violet}{中国海洋大学~~信息科学与工程学院}}
\date{2013~年~10~月~11~日}
%\titlegraphic{\vspace{-6em}\includegraphics[height=7cm]{ouc}\vspace{-6em}}
\frame{ \titlepage }
%%----------------------------------------------------------
\section*{目录}
\frame{\frametitle{目录}\tableofcontents}
%%----------------------------------------------------------
\section{LU and Cholesky Factorizations}
\subsection{Easy-to-Solve Systems}%如果你想书签不出现问题,请不要用\XeTeX
                                 %这类复杂的指令,直接写XeTeX吧
\begin{frame}
let us consider a system of $n$ unknows $x_{1}$, $x_{2}$, ..., $x_{n}$. It can be written in the form 
%\begin{displaymath}
%\left[ \begin{array}{ccc}
%a_{11} & a_{12} & ... & a_{1n}\\
%a_{21} & a_{22} & ... & a_{2n}\\
%a_{31} & a_{32} & ... & a_{3n}\\
%\vdots & \vdots & \ddots & \vdots\\
%a_{n1} & a_{n2} & ... & a_{nn}\\
%\end{array} \right]*\left[ \begin{array} x_{1}\\x_{2}\\x_{3}\\ \ldots\\x_{n}\\ \end{array} \right]=\left[ \begin{array} b_{1}\\b_{2}\\b_{3}\\ \ldots\%%\b_{n}\\ \end{array} \right]
%\end{displaymath}

\begin{displaymath}
\left[ \begin{array}{cccc}
a_{11} & a_{12} & \ldots & a_{1n}\\
a_{21} & a_{22} & \ldots & a_{2n}\\
a_{31} & a_{32} & \ldots & a_{3n}\\
\vdots & \vdots & \ddots & \vdots \\
a_{n1} & a_{n2} & \ldots & a_{nn}
\end{array} \right]\left[ \begin{array}{cccc} x_{1}\\x_{2}\\x_{3}\\ \vdots\\x_{n}\\ \end{array} \right]=\left[ \begin{array}{cccc} b_{1}\\b_{2}\\b_{3}\\ \vdots\\b_{n}\\ \end{array} \right]
\end{displaymath}  
The matrices in this equation are denoted by $A$, $x$, and $b$. Thus, our system is simply $Ax=b$
\end{frame}         
% \XeTeXpicfile "./logo.jpg" xscaled 100 yscaled 100 %插图也没有问



\begin{frame}

Matrix $A$ has a \textbf{diagonal structure}. This mains that all the nonzero elements of $A$ are on the main diagonal.
\begin{displaymath}
\left[ \begin{array}{ccccc}
a_{11} & 0 & 0 &\ldots & 0 \\
0 & a_{22} & 0 &\ldots & 0 \\
0 & 0 & a_{33} & \ldots & 0\\
\vdots & \vdots & \vdots & \ddots & \vdots \\
0 & 0 & 0 & \ldots & a_{nn}
\end{array} \right]\left[ \begin{array}{cccc} x_{1}\\x_{2}\\x_{3}\\ \vdots\\x_{n}\\ \end{array} \right]=\left[ \begin{array}{cccc} b_{1}\\b_{2}\\b_{3}\\ \vdots\\b_{n}\\ \end{array} \right]
\end{displaymath}
The solution is 
\begin{displaymath}
x=\left[ \begin{array}{cccc} b_{1}/a_{11}\\b_{2}/a_{22}\\b_{3}/a_{33}\\ \vdots\\b_{n}/a_{nn}\\ \end{array} \right]
\end{displaymath}
\end{frame}


\begin{frame}
\frametitle{lower triangular structure}
We assume a \textbf{lower triangular structure} for $A$. This means that all the nonzero elements of $A$ are situated on or below the main diagoal, the system is
\begin{displaymath}
\left[ \begin{array}{ccccc}
a_{11} & 0 & 0 &\ldots & 0 \\
a_{21} & a_{22} & 0 &\ldots & 0 \\
a_{31} & a_{32} & a_{33} & \ldots & 0\\
\vdots & \vdots & \vdots & \ddots & \vdots \\
a_{n1} & a_{n2} & a_{n3} & \ldots & a_{nn}
\end{array} \right]\left[ \begin{array}{cccc} x_{1}\\x_{2}\\x_{3}\\ \vdots\\x_{n}\\ \end{array} \right]=\left[ \begin{array}{cccc} b_{1}\\b_{2}\\b_{3}\\ \vdots\\b_{n}\\ \end{array} \right]
\end{displaymath}
To solve this, assume that $a_{ii}\neq0$ for all $i$; then obtain $x_{1}$ from the first equation. With the value of $x_{1}$ substituted into the second equation, solve the second equation for $x_{2}$. We proceed in the same way, obtaining $x_{1}$, $x_{2}$, ..., $x_{n}$, one at a time and in this order. A formal argorithm for the solution in this case is called \textbf{forward substitution}:
\end{frame}

\begin{frame}
input $n$, $(a_{ij})$, $(b_{i})$ \\
for $i=1$ to $n$ do\\
$x_{i}\leftarrow(b_{i}-\sum_{j=1}^{i-1}a_{ij}x_{j})/a_{ii}$\\
end do\\
output $(x_{i})$\\
As is customary, any sum of the type $\sum_{j=\alpha}^{\beta}x_{i}$ in which $\beta<\alpha$ is interpreted to be $0$.
\end{frame}

\begin{frame}
\frametitle{EXAMPLE 1}

\begin{displaymath}
\left[ \begin{array}{cccc}
-1 & 0 & 0 & 0 \\
-3 & 1 & 0 & 0 \\
-2 & 2 & 1 & 0 \\
 3 & 0 & -1 & 1 \\
\end{array} \right]\left[ \begin{array}{cccc} x_{1}\\x_{2}\\x_{3}\\ x_{4}\\ \end{array} \right]=\left[ \begin{array}{cccc} 0\\ -1 \\ 0 \\ -1 \\ \end{array} \right]
\end{displaymath}
\end{frame}

\begin{frame}
\frametitle{EXAMPLE 1}
\lstinputlisting{./lowertriangular/lowerang.m}
\lstinputlisting{./lowertriangular/lowerangtest.m}
\end{frame}

\begin{frame}
\frametitle{upper triangular structure}
\textbf {Upper triangular} structure has the form
\begin{displaymath}
\left[ \begin{array}{ccccc}
a_{11} & a_{12} & a_{13} &\ldots & a_{1n} \\
0 & a_{22} & a_{23} &\ldots & a_{2n} \\
0 & 0 & a_{33} & \ldots & a_{3n}\\
\vdots & \vdots & \vdots & \ddots & \vdots \\
0 & 0 & 0 & \ldots & a_{nn}
\end{array} \right]\left[ \begin{array}{cccc} x_{1}\\x_{2}\\x_{3}\\ \vdots\\x_{n}\\ \end{array} \right]=\left[ \begin{array}{cccc} b_{1}\\b_{2}\\b_{3}\\ \vdots\\b_{n}\\ \end{array} \right]
\end{displaymath}
It must be assumed that $a_{ii}\neq0$ for $1\leq i \leq n$. The formal algorithm to solve for $x$ is as follows and is called \textbf{back substitution}.\\
input $n$, $(a_{ij})$, $(b_{i})$ \\
for $i=n$ to $1$ step -1 do\\
$x_{i}\leftarrow(b_{i}-\sum_{j=i+1}^{n}a_{ij}x_{j})/a_{ii}$\\
end do\\
output $(x_{i})$\\ 
\end{frame}


\begin{frame}
\frametitle{EXAMPLE 2}
\begin{displaymath}
\left[ \begin{array}{cccc}
3 & 0 & -1 & 1 \\
0 & -2 & 2 & 1 \\
0 & 0 & -3 & 1 \\
0 & 0 & 0 & 1 \\
\end{array} \right]\left[ \begin{array}{cccc} x_{1}\\x_{2}\\x_{3}\\ x_{4}\\ \end{array} \right]=\left[ \begin{array}{cccc} 0\\ -1 \\ 0 \\ -1 \\ \end{array} \right]
\end{displaymath}
\end{frame}

\begin{frame}
\frametitle{EXAMPLE 2}
\lstinputlisting{./uppertriangular/upperang.m}
\lstinputlisting{./uppertriangular/upperangtest.m}
\end{frame}


\begin{frame}
\begin{displaymath}
\left[ \begin{array}{cccc}
a_{11}& a_{12} & 0 \\
a_{21} & a_{22} & a_{23}\\
a_{31} & 0 & 0 \\
\end{array} \right]\left[ \begin{array}{cccc} x_{1}\\x_{2}\\x_{3}\\  \end{array} \right]=\left[ \begin{array}{cccc} b_{1}\\ b_{2} \\ b_{3} \\  \end{array} \right]
\end{displaymath}
If we simply reorder these equations, we can get a lower trianglar system:
\begin{displaymath}
\left[ \begin{array}{cccc}
a_{31}& 0 & 0 \\
a_{11} & a_{12} & 0\\
a_{21} & a_{22} & a_{23} \\
\end{array} \right]\left[ \begin{array}{cccc} x_{1}\\x_{2}\\x_{3}\\  \end{array} \right]=\left[ \begin{array}{cccc} b_{3}\\ b_{1} \\ b_{2} \\  \end{array} \right]
\end{displaymath}
Putting this another way, we should solve the equations of System not in the usual order $1$, $2$, $3$, but in the order $3$, $1$, $2$. 
\end{frame}


\begin{frame}
We wish to say that one row of $A$, say row $p_{1}$, has zero in positions $2$, $3$, ...,$n$. Then another row $p_{2}$, has zeros in positions $3$, $4$, ...,$n$, and so on. Let's assume that the \textbf{permulation vector} $(p_{1}, p_{2}, ...,p_{n})$ is known or has been determined somehow beforehand. Modifying our pervious algorithms, we arrive at \textbf{forward substitution} for a \textbf{permuted lower triangular system}:
\newline
input $n$, $(a_{ij})$, $(b_{i})$, $(p_{i})$\\
for $i=1$ to $n$ do\\
$x_{i}\leftarrow(b_{p_{i}}-\sum_{j=1}^{i-1}a_{p_{i}j}x_{j})/a_{p_{i}i}$\\
end do\\
output$(x_{i})$
\end{frame}


\begin{frame}
\frametitle{EXAMPLE 2}
\begin{displaymath}
\left[ \begin{array}{cccc}
1 & 0 & 0 & 0 \\
-2 & 2 & 1 & 0 \\
3 & 0 & -1 & 1 \\
-3 & 1 & 0 & 0 \\
\end{array} \right]\left[ \begin{array}{cccc} x_{1}\\x_{2}\\x_{3}\\ x_{4}\\ \end{array} \right]=\left[ \begin{array}{cccc} 0\\ 0 \\ -1 \\ -1 \\ \end{array} \right]
\end{displaymath}
\end{frame}

\begin{frame}
\lstinputlisting{./luanlowertriangular/lowertriangular.m}
\lstinputlisting{./luanlowertriangular/lowertriangulartest.m}
\end{frame}


\begin{frame}
Similarly,\textbf{back sustitution} for a \textbf{permuted upper triangular system} is as follows:\\
input $n$, $(a_{ij})$, $(b_{i})$, $(p_{i})$\\
for $i={n}$ to $1$ step $-1$ do\\
$x_{i}\leftarrow(b_{p_{i}}-\sum_{j=i+1}^{n}a_{p_{i}j}x_{j})/a_{p_{i}i}$\\
end do\\
output$(x_{i})$
\end{frame}


\subsection{LU-factorization}
\begin{frame}
\begin{itemize}
\item Suppose that $A$ can be factored into the product of a lower triangular matrix $L$ and an upper triangular matrix $U$: $A=LU$.
\item Then to solve the system of equations $Ax=b$,\\
$Lz=b$ solve for $z$\\
$Ux=z$ solve for $x$
\end{itemize}
\end{frame}



\begin{frame}
We begin with an $n\times n$ matrix $A$ and search for matrices
\begin{displaymath}
L=\left[ \begin{array}{ccccc}
l_{11} & 0 & 0 &\ldots & 0 \\
l_{21} & l_{22} & 0 &\ldots & 0 \\
l_{31} & l_{32} & l_{33} & \ldots & 0\\
\vdots & \vdots & \vdots & \ddots & \vdots \\
l_{n1} & l_{n2} & l_{n3} & \ldots & l_{nn}
\end{array} \right]
\end{displaymath}
\begin{displaymath}
U=\left[ \begin{array}{ccccc}
u_{11} & u_{12} & u_{13} &\ldots & u_{1n} \\
0 & u_{22} & u_{23} &\ldots & u_{2n} \\
0 & 0 & u_{33} & \ldots & u_{3n}\\
\vdots & \vdots & \vdots & \ddots & \vdots \\
0 & 0 & 0 & \ldots & u_{nn}
\end{array} \right]
\end{displaymath}
Such that
\begin{displaymath}
 A=LU
\end{displaymath}
\end{frame}

\begin{frame}
\begin{itemize}
\item We start with the formula for matrix multiplication: 
\begin{displaymath}
a_{ij}=\sum_{k=1}^n l_{is}u_{sj}=\sum_{k=1}^{min(i,j)} l_{is}u_{sj}
\end{displaymath}
\item If $u_{kk}$ or $l_{kk}$ has specified, we can use the Equation $a_{kk}=\sum_{s=1}^{k-1} l_{ks}u_{sk}+l_{kk}u_{kk}$ to detetermine the other.
\item If $l_{kk}\neq0$ Equation(1) can be used to obtain the elements $u_{kj}$,\\
 If $u_{kk}\neq0$ Equation(2) can be used to obtain the elements $l_{ik}$.
\begin{equation}
a_{kj}=\sum_{s=1}^{k-1} l_{ks}u_{sj}+l_{kk}u_{kj} \qquad   (k+1 \leq j \leq n)
\end{equation}
\begin{equation}
a_{ik}=\sum_{s=1}^{k-1} l_{is}u_{sk}+l_{ik}u_{kk} \qquad   (k+1 \leq i \leq n)
\end{equation}
\end{itemize}
\end{frame}






%\subsection{Simple Roots}
\begin{frame}
\frametitle{general LU-factorization}
input $n$, $(a_{ij})$\\
for $k=1$ to $n$ do\\
Specify a nonzero value for either $l_{kk}$ or $u_{kk}$ and compute the other form $l_{kk}u_{kk}=a_{kk}-\sum_{s=1}^{k-1}l_{ks}u_{sk}$\\
for $j=k+1$ to $n$ do\\
$u_{kj}\leftarrow(a_{kj}-\sum_{s=1}^{k-1}l_{ks}u_{sj})/l_{kk}$\\
end do\\
for $i=k+1$ to $n$ do\\
$l_{ik}\leftarrow(a_{ik}-\sum_{s=1}^{k-1}l_{is}u_{sk})/u_{kk}$\\
end do\\
end do\\
output $(l_{ij})$, $(u_{ij})$
\end{frame}

\begin{frame}
\frametitle{three factorizations}
\begin{itemize}
\item When $L$ is  unit lower triangular $(l_{ii}=1)$ for  $(1 \leq i \leq n)$, the algorithm is called \textbf{Doolittle's factorization}.
\item When $U$ is  unit upper triangular $(u_{ii}=1)$ for  $(1 \leq i \leq n)$, the algorithm is called \textbf{Crout's factorization}.
\item When $U=L^{T}$ so that $l_{ii}=u_{ii}$ for $(1 \leq i \leq n)$, the algorithm is called \textbf{Cholesky's factorization}.
\end{itemize}
\end{frame}

\begin{frame}
\frametitle{EXAMPLE 3}
Find the Doolittle, Crout, and Cholesky factorizations of the matrix
\begin{displaymath}
A=\left[ \begin{array}{cccc}
60 & 30 & 20  \\
30 & 20 & 15  \\
20 & 15 & 12  \\
\end{array} \right]
\end{displaymath}
\lstinputlisting{./doolittle/doolittletest.m}
\end{frame}

\begin{frame}
\lstinputlisting{./doolittle/doolittle.m}
\end{frame}

\begin{frame}
\lstinputlisting{./Crout/crout.m}
%\lstinputlisting{./Crout/crouttest.m}
\end{frame}

\begin{frame}
The algorithm for the \textbf{Cholesky factorization} will then be as follows:
input $n$, $(a_{ij})$\\
for $k=1$ to $n$ do\\
$l_{kk}\leftarrow(a_{kk}-\sum_{s=1}^{k-1}l_{ks}^{2})^{1/2}$\\
for $i=k+1$ to $n$ do\\
$l_{ik}\leftarrow(a_{ik}-\sum_{s=1}^{k-1}l_{is}l_{ks})/l_{kk}$\\
end do\\
end do\\
output $(l_{ij})$
\end{frame}

\begin{frame}
\lstinputlisting{./Cholesky/cholesky.m}
%\lstinputlisting{./Cholesky/choleskytest.m}
\end{frame}



\begin{frame}
\theoremstyle{plain}
\newtheorem{theorem}{THEOREM}
\begin{theorem}[Theorem on LU-Decompostition]
 If all $n$ leading principal minors of the $n\times n$ matrix $A$ are nonsigular, then $A$ has LU-decompostition.
\end{theorem} 
\begin{proof}
Recall that the $k$th leading \textbf{principal minor} of the matrix $A$ is the matrix
\begin{displaymath}
A_{k}=\left[ \begin{array}{ccccc}
a_{11} & a_{12} & \ldots & a_{1k} \\
a_{21} & a_{22} & \ldots & a_{2k} \\
\vdots & \vdots & \ddots & \vdots \\
a_{k1} & a_{k2} & \ldots & a_{kk}
\end{array} \right]
\end{displaymath}
Suppose that $L_{k-1}$ and $U_{k-1}$ have been obtained. Hence, Equation $a_{ij}=\sum_{k=1}^n l_{is}u_{sj}=\sum_{k=1}^{min(i,j)} l_{is}u_{sj}$ states that $A_{k-1}=L_{k-1}U_{k-1}$.\\
Since $A_{k-1}$ is nonsigular by hypothesis, $L_{k-1}$ and $U_{k-1}$ are also nonsinglar.
\end{proof}
\end{frame}


\begin{frame}
\begin{proof}
Since $L_{k-1}$ is nonsigular, we can solve the system $\sum_{s=1}^{k-1}l_{is}u_{sk}=a_{ik}$  $(1 \leq i \leq k-1)$ for the quantities $u_{sk}$ with $(1 \leq s \leq k-1)$. These elements lie in the $k$th column of $U$.\\
Since $U_{k-1}$ is nonsigular, We can solve the system $\sum_{s=1}^{k-1}l_{ks}u_{sj}=a_{kj}$ $(1 \leq j \leq k-1)$ for the quantities $l_{ks}$ with $(1 \leq s \leq k-1)$. These elements lie in the $k$th row of $L$.\\
From the requirement $a_{kk}=\sum_{s=1}^{k-1} l_{ks}u_{sk}+l_{kk}u_{kk}$ we can obtain $u_{kk}$ since $l_{kk}$ has been specified as unity. \\
Thus, all the new elements necessary to form $L_{k}$ and $U_{k}$ have been defined.\\
The induction is completed by nothing that $l_{11}u_{11}=a_{11}$ and, therefore, $l_{11}=1$ and $u_{11}=a_{11}$.
\end{proof}
\end{frame}


\begin{frame}
\begin{theorem}[Cholesky Theorem on $LL^{T}$-Factorization]
 If $A$ is a real, symmetric, and positive definite matrix, then it has a unique factorization, $A=LL^{T}$, in which $L$ is lower triangular with a positive diagonal.
\end{theorem} 
\begin{proof}
A matrix $A$ is symmertic and positive definite if $A=A^{T}$ and $x^{T}Ax>0$ for every nonzero vector $x$. By considering special vectors of the form $x=(x_{1}, x_{2}, x_{3}, ,..., x_{k}, 0, 0, ...,0)^{T}$, we see that the leading principal minors of $A$ are positive definite. Therom $1$ implies that $A$ has LU-decomposition. By the symmetry of $A$, we have $LU=A=A^{T}=U^{T}L^{T}$,This implies that $U(L^{T})^{-1}=L^{-1}U^{T}$. The left of the equation is upper trigular, whereas the right member is lower triangular. There is a diagoal matrix $D$ such $D=U(L^{T})^{-1}$. $U=DL^{T}$ and $A=LDL^{T}$. $D$ is positive definite. $A=L'L'^{T}$, where $L'=LD^{1/2}$.
\end{proof}
\end{frame}

\begin{frame}
\begin{proof}
Uniqueness proof:\\
Suppose that there exist $L_{1}$ such that $A=LL^{T}=L_{1}L_{1}^{T}$. Let's show that $L=L_{1}$.\\ 
$A=LL^{T}=L_{1}L_{1}^{T}$\\
$L^{-1}LL^{T}=L^{-1}L_{1}L_{1}^{T}$\\
$L^{T}=L^{-1}L_{1}L_{1}^{T}$\\
$L^{T}(L_{1}^{T})^{-1}=L^{-1}L_{1}=D^{*}$\\
$L_{1}=LD^{*}$\\
$L^{T}=D^{*}L_{1}^{T}$\\
$D^{*}(LD^{*})^{T}=D^{*}{D^{*}}^{T}L^{T}=(D^{*})^{2}L^{T}$\\
$D^{*}(LD^{*})^{T}=D^{*}L_{1}^{T}=L^{T}$\\
$(D^{*})^{2}L^{T}=L^{T}$\\
$(D^{*})^{2}=I$\\
$D^{*}=I=L^{-1}L_{1}$\\
$L=L_{1}$\\
So it has a unique factorization. 
\end{proof}
\end{frame}





\end{document}
